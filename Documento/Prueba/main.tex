\documentclass[12pt]{article}

% Paquetes útiles
\usepackage[utf8]{inputenc}
\usepackage[spanish]{babel}
\usepackage{amsmath}
\usepackage{graphicx}

% Información del documento
\title{Mi Primer Documento en \LaTeX}
\author{Tu Nombre}
\date{\today}

\begin{document}

\maketitle

\section{Introducción}

Este es un documento básico en \LaTeX. Puedes escribir texto normal y \LaTeX{} se encargará del formato automáticamente.

\section{Listas}

Aquí hay una lista numerada:
\begin{enumerate}
    \item Primer elemento
    \item Segundo elemento
    \item Tercer elemento
\end{enumerate}

Y una lista con viñetas:
\begin{itemize}
    \item Elemento A
    \item Elemento B
    \item Elemento C
\end{itemize}


\section{Matemáticas}

LaTeX es excelente para escribir ecuaciones. Por ejemplo, la fórmula cuadrática es:

\[
x = \frac{-b \pm \sqrt{b^2 - 4ac}}{2a}
\]

También puedes escribir matemáticas en línea como $E = mc^2$ dentro del texto.

\section{Conclusión}

Este es un ejemplo básico para comenzar con \LaTeX. ¡Experimenta modificando el código!

\end{document}