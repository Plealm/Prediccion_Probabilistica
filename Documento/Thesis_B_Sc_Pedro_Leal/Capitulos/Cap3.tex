\chapter{Correcciones espinoriales} \label{cap:3}
En este último capítulo se presentarán las correcciones hiperfinas a este proceso, agregando la contribución spin-spin por medio de teoría de perturbaciones y minimizando nuevamente la función de la energía con respecto a las parámetros \(\alpha\),  \(\sigma\) y \(Q\) en los estados S, donde el intervalo de la energía base es coherente. Finalmente, se realiza el mismo proceso para las correcciones Spin-orbita  y Tensorial siendo una aproximación para estados \(l >0\) y encontrando los mismos parámetros anteriormente mencionados.

\section{Correcciones hiperfinas}
Las correcciones a usar deducidas en el Anexo \ref{AnexoCorrec} son:
\begin{align}
    V_{SS}^{Vectorial}(r) &= \frac{8 \alpha_s}{9m^2_Qr^2}(S_1 \cdot S_2)\delta(r)\nonumber\\
    V_{LS}(r)^{Vectorial} &= \frac{2 \alpha_s}{m^2_Qr^3}(L \cdot S) \label{eq: correc}\\
    V_{LS}(r)^{Escalar} &= -\frac{\sigma}{2m^2_Qr}(L \cdot S)\nonumber\\
    V_{T}(r)^{Vectorial} &=\frac{\alpha_s}{3m^2_Qr^2}S_{12}\nonumber
\end{align}

Con \(S = S_1 +S_2\), L es el operador momentum angular y \(S_{12}\) es el operador para la contribución tensorial. Los valores esperados de los operadores 
presentados, son

\begin{align*}
    \langle S_1 \cdot S_2 \rangle& = \frac{1}{2}s(s+1) - \frac{3}{4}\\
    \langle L \cdot S \rangle& =\frac{1}{2}[j(j+1) - l(l+1) - s(s+1)]\\
    \langle S_{12} \rangle& = \frac{2[2\langle S^2 \rangle\langle{L^2}\rangle - 3\langle{L \cdot S}\rangle -6\langle{L \cdot S}\rangle^2]}{(2l + 3)(2l + 1)}
\end{align*}

Donde \(j = l + s\), J es el operador angular total, s y l son los correspondientes números cuanticos del estado espectroscópico \(Q\overline{Q}\).

\section{Teoría de perturbaciones}
Las correcciones definidas se agregan a la energía base por medio de la teoría de perturbaciones\cite{libro},

\begin{align}
    E^1_n = \langle \psi_n^0|H'| \psi^0_n\rangle
\end{align}

Donde \(H'\) es el potencial correctivo, \(\psi^0_n\)
es la función de onda \ref{eq: solradial}, que se usará para \(l > 0\), encontrando de manera 
genérica la integral,

\begin{align}
    \int_0^\infty z^{m} [A_i(a + z b)]^2 dz \label{eq: integral}
\end{align}

Tomando \(l = 1\), la potencia m tomará los siguientes valores,
\begin{align*}
    \begin{array}{lccc}
         V_{SS}^{\text{Vectorial}} &  {2l = m} & m = 2\\
         \\V_{LS}^{\text{Vectorial}} =  V_{T}^{Vectorial} &  {2l - 1 = m}& m = 1\\
         \\ V_{LS}^{\text{Escalar}} &{2l + 1 = m} & m = 3
    \end{array}
\end{align*}

En el caso de \(l = 0\), la única corrección que se puede realizar es
del spin-spin, por lo que usando \ref{eq: solradial} se obtendría 
\(m = -2\), pero la solución de la integral está definida para \(\Re(m)>-1\),
por lo que se toma la expresión para \(l = 1\) y se obtendría \(m = 0\).\\


Esta integral se obtiene paso a paso en el apéndice \ref{AnexoB}, pero sus resultados son:

\begin{align}
    \int_0^\infty
    z^n[Ai(a + z b)]^2 dz = \sum_{k=0}^n\frac{W_3[a, b, n] k! (n - k)!}{n! a^{n-k}b^k} - \sum_{k=0}^{n-1}\int_0^\infty
    z^k[Ai(a + z b)]^2 dz \label{eq: integral_1}
\end{align}

En el caso de la corrección spin-spin \ref{eq: correc} se toma la 
delta de Dirac como la gaussiana \(\delta(r) = \left(\frac{Q}{\sqrt{\pi}}\right)^3 e^{-Q^3r^2}\)\cite{delta}, permitiendo encontrar su integral usando series de Taylor y tomando \ref{eq: integral_1} como \(W_4(m + 2m',a,b)\),
\begin{align}
    \int_0^\infty z^m[Ai(a + zb)]^2\left(\frac{Q}{\sqrt{\pi}}\right)^3 e^{-Q^3z^2}dz = \sum^5_{m' = 0} \left(\frac{Q^{3(1+ m')}}{\pi^\frac{3}{2}m'!}\right)W_4(m + 2m',a,b)
\end{align}

Los valores numéricos encontrados se encuentran en el apéndice \ref{Anexonum}.


\section{Ajuste con las correcciones Spin-Spin}
Agregando las contribuciones dadas por la corrección spin-spin
se puede diferencias los estados \(\eta_c (1^1S_1)\) donde su
momento angular orbital es \(l = 0\) con un spin \(S = 0\) y estados S
con spin \(S = 1\) \(j/\psi(1^3S_1)\). Asumiendo los valores de las 
masas según el PDG \cite{PDG}, se realiza el ajuste para encontrar 
los parámetros \(\alpha\) constante de acople y coeficiente del termino $\frac{1}{r}$ en el potencial, \(\sigma\) coeficiente del termino
lineal del potencial y Q parámetro de la gaussiana usada para expresar
la delta de Dirac. La masa a miminizar es

\begin{align}
    M^1 = M + E^1_{\text{spin-spin}}.
\end{align}

\subsection{Charmonium}
Para el Charm se asigna la masa \(\SI{1.27\pm 0.02}{\giga\electronvolt}\), con está se encuentra el ajuste

\begin{table}[H]
\resizebox{13cm}{!}{
\begin{tabular}{@{}ccccccc@{}}
\toprule
\textbf{} & Este trabajo[GeV]    &           Expt  \cite{PDG} [GeV]   &  Lattice QCD \cite{lattice}[GeV]& \cite{data} [GeV] \\ \midrule
$M_{\eta_c}(1s)$      &3.17$\pm$0.13    & 2.9839$\pm$0.0004 & 3.2924(9)   & 2.981 \\
$M_{\eta_c}(2s)$      &3.39$\pm$0.18    & 3.6375$\pm$0.0011 & 4.24(6)       &3.635       \\
$M_{\eta_c}(3s)$      &3.59$\pm$0.22     &                   &                     &3.989  \\
$M_{\eta_c}(4s)$      &3.77$\pm$0.26    &                   &                     & 4.401 \\ 
$M_{j/\psi}(1s)$   &3.17$\pm$0.13       & 3.096900$\pm$0.006& 3.4452(16)     & 3.096  \\ 
$M_{\psi}(2s)$   &3.39$\pm$0.18         & 3.68610$\pm$0.06  & 4.35(7)     & 3.685  \\ 
$M_{\psi}(3s)$   &3.59$\pm$0.22       & 4.039$\pm$0.001   &                     &4.039 \\
$M_{\psi}(4s)$   &3.77$\pm$0.26         & 4.421$\pm$ 0.0004  &                    & 4.427 \\\bottomrule
\end{tabular}
\caption{Resultados de las masa del Charmonium para los estados S con diferenciación de spin.}
\label{table: charm_spin}
}
\end{table}

Los valores encontrados en la tabla \ref{table: charm_spin} poseen diferencias entre los estados con y sin spin de ordenes \(1 \times 10^{-8}[GeV]\) que con sus respectivas incertidumbres no es notorio. Los valores de masa fueron dados usando los siguientes parámetros encontrados por medio de una minimización \(\chi^2 = 4337.4\)
\begin{equation}
\boxed{
\begin{array}{rcl}
 \sigma &= \SI{0.068\pm0.022}{\giga\electronvolt\per\femto\meter}\\
 \alpha_s &= 0.4987(8) \\
 Q &= 0.010(6)
\end{array}
}
\end{equation}

\subsection{Bottomonium}
De igual manera el bottom posee una masa \(m_b=\SI{4.18\pm 0.04}{\giga\electronvolt}\), usando esta masa se pueden encontrar los siguientes valores

\begin{table}[H]
\resizebox{13cm}{!}{
\begin{tabular}{@{}ccccccc@{}}
\toprule
\textbf{} & Este trabajo [GeV]   &           Expt  \cite{PDG}  [GeV]  &  Lattice QCD \cite{lattice}[GeV]& \cite{data}  [GeV]\\ \midrule
$M_{\eta_b}(1s)$      &9.54$\pm$0.04    & 9.3987$\pm$0.0020    & 7.3776(8)   & 9.398 \\
$M_{\eta_b}(2s)$      &9.96$\pm$0.05    & 9.999$\pm$0.004      & 8.202(5)       &9.990       \\
$M_{\eta_b}(3s)$      &10.32$\pm$0.07   &                      &                     &10.329  \\
$M_{\eta_b}(4s)$      &10.66$\pm$0.08   &                      &                     & 10.573 \\ 
$M_{\Upsilon}(1s)$   &9.54$\pm$0.04     & 9.46030$\pm$0.00026  & 7.4100(9)     & 9.460  \\ 
$M_{\Upsilon}(2s)$   &9.96$\pm$0.05     & 10.02326$\pm$0.00031 & 8.209(5)     & 10.023  \\ 
$M_{\Upsilon}(3s)$   &10.32$\pm$0.07    & 10.3552$\pm$0.0005   &                     &10.355 \\
$M_{\Upsilon}(4s)$   &10.66$\pm$0.08    & 10.5794$\pm$ 0.0012  &                    & 10.586 \\\bottomrule
\end{tabular}
\caption{Resultados de las masa del Bottomonium para los estados S con diferenciación de spin.}
\label{table: bottom_spin}
}
\end{table}

Los valores encontrados en la tabla \ref{table: bottom_spin} poseen diferencias entre los estados con y sin spin de ordenes \(1 \times 10^{-7}\) que con sus respectivas incertidumbres no es notorio. Los valores de masa fueron dados usando los siguientes parámetros encontrados por medio de una minimización \(\chi^2 = 54.1\)
\begin{equation}
\boxed{
\begin{array}{rcl}
 \sigma &= \SI{0.31\pm0.02}{\giga\electronvolt\per\femto\meter}\\
 \alpha_s &= 0.140(5) \\
 Q &= 0.10(2)
\end{array}
}
\end{equation}

%%%%%%%%%%%%%%%%%%%%%%%%%%%%%%%%%%%%%%%%%%%%%%%%%%%%%%%%%%%%%%%%%%%%%%%%%%%%%%%%%%%%%%%%%%%%%%%%%%%%%%%%%%%%%%%%%%%%%%%%%%%%%%%%%%%%%%%%%%%%%%%%%%%%%%%%%%%%%%%%%%%%%%%%%%%
\section{Ajuste para estados \(l> 0\)}
Se agregan los estados
\(\chi_{c0}(1^3P_0), \chi_{c1}(1^3P_1)\) y 
\(\chi_{c2}(1^3P_2)\).  En este caso \(M\) encuentra dependencia de
los tres parámetros y se realiza el ajuste  de todos los tres al tiempo,
encontrando una nueva masa:

\begin{align}
    M^1 = M + E^1_{\text{spin-spin}}+ E^1_{\text{spin-orbita}}+ E^1_{\text{tensorial}}
\end{align}

Cabe aclara que se encuentran dos condiciones muy importantes a tener en cuenta,
a la hora de tomar estados \(l>0\):
\begin{enumerate}
    \item Cualquier estado que sea \(l > 0\), pero \(S = 0\), poseerá
    la misma función que el estado \(\eta_c\) que le corresponda según
    su número cuántico n.
    
    \item Al expandir forzosamente este modelo a \(l >0 \) se está
    realizando una segunda aproximación, suponiendo una aproximación
    fuerte, \textbf{la energía base que no depende de l, supondremos que
    en primera aproximación puede describir estados mayores a l} y 
    posteriormente agregando está contribución l
    por medio de teoría de perturbaciones.
    
\end{enumerate}
\subsection{Charmonium}
Para la definición de la función \(\chi^2\) se utilizaron los datos descritos
en el capítulo \ref{cap:2}, los valores de los estados \(1P\) fueron tomados el PDG \cite{PDG}, obteniendo los resultados,

\begin{table}[H]
\resizebox{13cm}{!}{
\begin{tabular}{@{}ccccccc@{}}
\toprule
\textbf{} & Este trabajo  [GeV]  &  Expt  \cite{PDG}[GeV] &  Lattice QCD \cite{lattice}     [GeV]       & \cite{data}[GeV]  \\ \midrule
$M_{\eta_c}(1s)$      &4.21$\pm$0.35             & 2.9839$\pm$0.0004      & 3.2924(9)    & 2981 \\
$M_{\eta_c}(2s)$      &4.3$\pm$1.8               & 3.6375$\pm$0.0011      & 4.24(6)      & 3.635 \\
$M_{J/\psi}(1s)$      &4.22$\pm$0.34             & 3.096900$\pm$0.000006  & 3.4452(16)   & 3.096 \\
$M_{\psi}(2s)$        &4.97$\pm$0.14             & 3.68610$\pm$0.00006    & 4.35(7)      & 3.685 \\ 
$M_{\chi_{c0}}(1P)$   &2.7$\pm$1.1               & 3.41471$\pm$ 0.0003    & 4.079(12)    & 3.413 \\ 
$M_{\chi_{c1}}(1P)$   &3.2$\pm$1.2               & 3.51067$\pm$ 0.00005   & 4.052(89)    & 3.511  \\ 
$M_{\chi_{c2}}(1P)$   &5.2$\pm$0.7               & 3.55617$\pm$0.00007    & 4.272(15)    & 3.555 \\ \bottomrule
\end{tabular}
\caption{Resultados de las masa del Charmonium para los estados $\eta_c(1S)$, $\eta_c(2S)$, $\eta_c(3S)$, $\eta_c(4S)$, $\chi_{c0}$, $\chi_{c1}$ y $\chi_{c2}$ comparados con diferentes valores teóricos y experimentales.}
\label{table: 3}
}
\end{table}

Normalizando la incertidumbre encontrada por la raíz del valor del \(\chi^2 = 275040.1\)
en su mínimo, se encontraron los estados descritos en la tabla \ref{table: 3}.

La minimización, permite encontrar los parámetros,

\begin{equation}
\boxed{
\begin{array}{rcl}
 \sigma &= \SI{0.29\pm0.01}{\giga\electronvolt\per\femto\meter}\\
 \alpha_s &= 0.457(11)\\
 Q &= 0.468(24)
\end{array}
}
\end{equation}
\subsection{Bottomonium}

Tomando los datos para los estados P de \cite{PDG} y la masa \(\SI{4.18\pm0.04}{\giga\electronvolt}\)\cite{PDG}, se realiza nuevamente un ajuste, tal que, 
\begin{table}[H]
\resizebox{13cm}{!}{
\begin{tabular}{@{}ccccccc@{}}
\toprule
\textbf{}           & Este trabajo  [GeV]  & Expt  \cite{PDG}     [GeV]   &  Lattice QCD \cite{lattice}[GeV]& \cite{data} [GeV] \\ \midrule
$M_{\eta_b}(1s)$      & 9.76$\pm$0.04    & 9.3987$\pm$0.002     & 7.3776(8)   & 9.398 \\
$M_{\eta_b}(2s)$      &9.80$\pm$0.09     & 9.999$\pm$0.004      & 8.202(5)    & 9.990 \\
$M_{\Upsilon}(1s)$      &9.76$\pm$0.04   & 9.46030$\pm$0.00026  & 7.4100(9)   & 9.460 \\
$M_{\Upsilon}(2s)$      &9.81$\pm$0.05   & 10.02326$\pm$0.00031 & 8.209(5)    & 10.023 \\ 
$M_{\chi_{b0}}(1P)$   &7.8$\pm$2.1       &  9.85944$\pm$ 0.0005 & 8.220(8)     & 9.859 \\ 
$M_{\chi_{b1}}(1P)$   &8.7$\pm$1.1       &  9.89278$\pm$ 0.0004 & 8.243(8)     & 9.892  \\ 
$M_{\chi_{b2}}(1P)$   &10.8$\pm$1.0      &  9.91221$\pm$0.0004  & 8.274(9)     & 9.912 \\ \bottomrule
\end{tabular}
\caption{Resultados de las masas del Bottomonium con las mismas características que con el Charmonium \ref{table: 3}.}
\label{table: 4}
}
\end{table}


la anterior minimización da como resultado los parámetros,

\begin{equation}
\boxed{
\begin{array}{rcl}
 \sigma &= \SI{0.0163\pm0.0068}{\giga\electronvolt\per\femto\meter}\\
 \alpha_s &= 0.197(1) \\
 Q &= 0.188(3)
\end{array}
}
\end{equation}

\section{Conclusión}

El modelos utilizado posee grandes limitaciones, observando que
solo puede ser aplicable satisfactoriamente en estados con \( l = 0\) (S). También se observa como al introducir las correcciones Spin-Spin no se encuentra
una variación apreciable de la masa encontrada para los estados \(1^1S_1\) y
\(1^3S_1\) sino correcciones de ordenes \(1 \times 10^{-7}\) con los parámetros
hallados por medio de la minimización, esto puede ser explicable ya que
en la deducción de la integral también se realizan ciertas aproximaciones como la
serie de Taylor hasta un \(n = 5\) y el binomio de Newton.

\section{Futuras contribuciones}

Para mejorar los resultados presentados en este trabajo, es necesario resolver
la ecuación de Schrödinger, tal que permita cuantizar la energía para diferentes
niveles de \(l\), este proceso puede ser logrado usando un polinomio que 
describa el comportamiento entre las soluciones asintóticas encontrando
de esta manera una dependencia de la energía en función del momentum angular 
orbital.

La integración realizada para la teoría de perturbaciones carece de rigor
matemático, lo que al realizar análisis más profundos sobre esta
se pueden encontrar nuevas condiciones o términos correctivos. Por lo que se 
observa un nuevo camino usando integración numérica, reemplazando los parámetros
encontrados en cada caso y usando la expresión normalizada de teoría de perturbaciones

\begin{align*}
    E^1_{nl} = \frac{\langle \psi_{nl}| H^1|\psi_{nl}\rangle}
    {\langle \psi_{nl}|\psi_{nl}\rangle}
\end{align*}

se puede llegar a un resultado sin hacer uso de las expresiones aproximadas de la
integral, esperando mejorar los datos.
