\chapter*{Introducción}

La física de partículas cambió para siempre el 11 de Noviembre de 1974, cuando
en Brookhaven National Laboratory (BNL)\cite{BNL} y en Stanford
Linear Accelerator Center (SLAC) \cite{SLAC} se encontró una nueva resonancia \(J/\psi\), fue el primer estado observado de un nuevo sistema, por entonces desconocido, se llamaría el charmonium \(c\overline{c}\), un mesón quarkonico (la composición de un quark-antiquark pesado). Este nuevo sistema se puede entender como la analogía del  positronio con la esperanza de que fuese como el átomo de hidrógeno que ayudo a entender la física atómica, en este caso enfocado en la dinámica hadrónica. De igual manera en 1977 se descubrió el primer estado del Bottomonium\cite{B1977}.\\

La teoría física que describiría estos mesones es la QCD (Quantum Chromodynamics),
pero hay dos impedimentos para una deducción analítica: 1. la libertad asintótica \cite{freeas}
que para energías bajas la constante de acople \(\alpha_s\) posee valores 
considerables. 2. el confinamiento, que implica que los quarks siempre
se encuentran dentro de hadrones y no pueden verse de manera individual. Estas 
consideraciones hacen que no sea posible usar teoría de perturbaciones
en la QCD.\\

Como consecuencia de esta imposibilidad de realizar una deducción netamente
teorica, se propone un modelo no relativista, sugerido por las características del sistema (m \(\gg\) p). Lo que permite presentar
potenciales no-relativistas fenomenológicos\cite{modelcharm} y aplicarlos como herramientas para
ajustar el espectro quarkonico.\\

El objetivo de este trabajo de grado es construir un potencial con parámetros 
libres los cuales se ajustarán usando los datos provistos por el PDG con la 
adición de las correcciones espinoriales necesarias para realizar la debida 
minimización.\\

Este trabajo, se divide en: capítulo \ref{cap:1}, deducción del potencial de Cornell desde primeros principios; capítulo \ref{cap:2} Implementación de la ecuación de Schrödinger, además de resultados para los estados S; capítulo \ref{cap:3} Deducción 
de las correcciones espinoriales con su adición por medio de teoría de 
perturbaciones y  resultados finales.