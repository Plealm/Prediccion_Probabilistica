\chapter{Conceptos Básicos para Modelos de quarkonia} \label{cap:1}

Este capítulo describe la cinemática con la cual se define la ecuación de Schrödinger a solucionar, la deducción del potencial de Cornell
desde primeros principios y como se realiza todo este proceso desde la mecánica
cuántica no relativista.
\\

En principio, los estados ligados en una teoría de campos relativista se estudian con el formalismo de Bethe-Salpeter, pero es importante considerar las escalas características de los quark ligados que se profundizarán, con una constante de estructura de la QCD  \(\Lambda_{QCD} \approx \SI{0.2}{\giga\electronvolt}\)\cite{phdthesis}, el quark charm tiene una masa \(m_c \approx \SI{1.27}{\giga\electronvolt}\) y en el caso del quark bottom \(m_b \approx \SI{4.18}{\giga\electronvolt}\)\cite{PDG} . Por lo tanto, se puede analizar estos procesos desde una teoría \textit{no relativista} y así mismo ser tratado desde
la solución de la ecuación de Schrödinger.

\section{Estados espectroscópicos}

Los estados espectroscópicos son los posibles estados cuánticos en los que
se puede encontrar un sistema cuántico, en este caso el quarkonia, como
el charmonium y el Bottomonium son fermiones de spin \(\frac{1}{2}\) los
posibles valores que puede tomar su spin conjunto \(S = S_1 + S_2\) se
denota como,

\begin{align*}
    0 = \Big|\frac{1}{2} - \frac{1}{2}\Big| \leq S \leq \Big|\frac{1}{2} + \frac{1}{2}\Big| =1
\end{align*}

Otro número cuántico fundamental es el momento angular orbital L que puede 
tomar diferentes valores siempre y cuando el sistema posea ese estado,



\begin{table}[H]
\begin{tabular}{cllcl}
\multirow{2}{*}{L =} & 0 & 1 & 2 & ... \\
                   & S & P & D & ...
\end{tabular}
\end{table}

Con estos dos números cuanticos se encuentra las relaciones para el intercambio
de carga y la paridad,
\begin{align*}
    C  &= (-1)^{L + S}\\
    P &= (-1)^(L + 1)
\end{align*}

Conociendo los valores esperados de la carga y paridad para el estado de interés se puede
expresar como,

\begin{align*}
    J^{PC}
\end{align*}

Donde J es el momento angular total. La notación que se usará en el desarrollo de este trabajo
es diferente constando de los números cuánticos,

\begin{align*}
    n^{2S + 1}L_J
\end{align*}

Algunos de los estados quarkonicos más representativos con su nomenclatura
espectroscópica,
\begin{table}[H]
\begin{tabular}{@{}ccc@{}}
\toprule
                & \(n^{2S + 1}L_J\) & $J^{PC}$ \\ \midrule
$\eta_c(1S)$    & $1^1S_1$          & $0^{-+}$ \\
$\eta_c(2S)$    & $2^1S_1$          & $0^{-+}$ \\
$J/\psi(1S)$    & $1^3S_1$          & $1^{--}$ \\
$\psi(2S)$      & $2^3S_1$          & $1^{--}$ \\
$\chi_{c0}(1P)$ & $1^3P_0$          & $0^{++}$ \\
$\chi_{c1}(1P)$ & $1^3P_1$          & $1^{++}$ \\
$\chi_{c2}(1P)$ & $1^3P_2$          & $2^{++}$ \\ \bottomrule
\end{tabular}
\caption{Representación de estados del charmonium en notación \(n^{2S + 1}L_J\) y $J^{PC}$}
\label{table: notacion}
\end{table}

La anterior tabla \ref{table: notacion} demuestra los estados del charmonium,
pero de igual manera existen los mismos estados para el Bottomonium, por lo que 
se cambia la c por b.

\section{Definición de la cinemática}

Teniendo presente que la ecuación de Schrödinger es

\begin{align*}
    H \psi = E\psi
\end{align*}
Para este caso, donde se encuentra un sistema de dos partículas ligadas 
de masas \(m_1\), \(m_2\) que interactúan por un potencial  \(V_0\),
su hamiltoniano se define 

\begin{align*}
    H = \frac{\Vec{p_1}^2}{2m_1} + \frac{\Vec{p_2}^2}{2m_2} + V_0(|\Vec{r_1} - \Vec{r_2}|)
\end{align*}

Por medio de transformaciones, se puede definir el problema desde el centro de
masa, quedando la ecuación de Schrödinger como

\begin{align}
    \left[-\frac{\nabla^2}{m_Q} + V_0(r)\right] \psi(r)= E\psi(r) \label{eq: schro}
\end{align}

Donde \(m_Q\) es la masa del quark en cuestión y \(V_0\) es un potencial genérico.

\section{Límite no relativista}

Con el límite no relativista la complejidad de este proceso se reduce
drásticamente, teniendo presente que el límite no relativista también se puede denotar como \(\mathbf{p \to 0}\), la expresiones relativistas se simplifican, en el caso de la energía,

\begin{align}
    E = \sqrt{p^2 + m^2} \to E = m,
\end{align}

los espinores de Dirac que describen fermiones de masa m,  cuadri-momentum p y spin \(\sigma\), también encuentran cambios,

\begin{align*}
    u(p, \sigma) = \sqrt{\left(\frac{E + m}{2 m}\right)}
    \begin{pmatrix}
    1 \\ \frac{\mathbf{\sigma \cdot p}}{E + m}
    \end{pmatrix} \chi_\sigma, \\
    v(p, \sigma) = \sqrt{\left(\frac{E + m}{2 m}\right)}
    \begin{pmatrix}
    \frac{\mathbf{\sigma \cdot p}}{E + m}\\ 1
    \end{pmatrix} \chi_\sigma^c,
\end{align*}
Pero con el límite la expresión anterior queda reducida a,
\begin{align}
    u(\sigma) = \begin{pmatrix}
    \chi_\sigma \\ 0
    \end{pmatrix},\qquad
    v(\sigma) = \begin{pmatrix}
    0 \\\chi_\sigma^c 
    \end{pmatrix}.\label{eq: operadores}
\end{align}

La normalización espinorial que se optará a usar en el presente análisis es:


\begin{align*}
    u^\dagger(p, \sigma)u(p, \tau) = v^\dagger(p, \sigma)v(p, \tau) = \frac{E}{m}\delta_{\sigma, \tau}
\end{align*}

que en el caso no relativista, se reescribe como,

\begin{align}
     u^\dagger(\sigma)u(\tau) = \chi_\sigma^\dagger \chi_\tau = \delta_{\sigma, \tau} \label{eq: norm}
\end{align}

Por medio de la anteriores relaciones se procederá a realizar los futuros cálculos
sobre las amplitudes de los procesos de interés.

\section{Potencial estático en QCD}

El potencial estático para la QCD se deriva de las posibles interacciones que se encuentran a nivel árbol para el proceso 
\begin{align}
    q_i(p_1, \sigma_1)\overline{q_j}(p_2, \sigma_2) \to q_k(q_1, \tau_1) \overline{q_l}(q_2, \tau_2)
\end{align}

tal que , los índices denotan los grados de libertad del color. El lagrangiano
que representa el acoplamiento con una fuerza \(g_s\) de un quark q, con el campo espinorial de Dirac \(q_i(x)\), en el campo gluónico \(G_a^\mu\)\cite{hiperfina},

\begin{align}
    \mathcal{L} = g_s \overline{q}_i (x)\gamma_\mu \frac{\lambda^a}{2} q_j(x) G_a^\mu(x)
\end{align}

Donde \(\frac{\lambda^a}{2}\) son las matrices de Gell-Mann que poseen la relación \(T^a = \frac{\lambda^a}{2}\), siendo \(T^a\)
los generadores del grupo SU(3). Es importante notar como
únicamente dos diagramas de Feynman contribuyen al orden más bajo en los procesos perturbativos, los cuales serán analizados 
en las siguientes subsecciones.
%%%%%%%%%%%%%%%%%%%%%%%%%%%%%%%%%%%%%%%%%%%%%%%%%%%%%%%%
\subsection{Intercambio Gluónico}
El primer proceso posee un diagrama de Feynman a analizar que  denota el intercambio gluónico entre un quark \(q_i(p_1, \sigma_1)\) y  un anti-quark \(\overline{q}_j(p_2, \sigma_2)\) representado en 
\begingroup
\begin{figure}[H]
\begin{center}
    \feynmandiagram [scale=2,transform shape][horizontal=a to b] {
i1 [particle=\(q_{i}\)] -- [fermion] a -- [fermion] i2 [particle=\(q_{k}\)],
a -- [gluon, momentum'=\(k\)] b,
f1 [particle=\(\overline{q}_l\)] -- [fermion] b -- [fermion] f2 [particle=\(\overline{q}_j\)],
};
\end{center}
\caption{Intercambio Gluónico.}
\label{fig: gluónico}
\end{figure}
\endgroup

Usando las leyes de Feynamn descritas en \ref{AnexoA} se puede encontrar que su contribución es:

\begin{align}
    \mathcal{M}_{gluon} &= -\frac{1}{k^2} \left[\overline{u}(q_1, \tau_1)
    g_s \gamma^\mu \frac{\lambda_{ki}^a}{2} u(p_1, \sigma_1)\right]
    \left[\overline{v}(q_2, \tau_2)
    g_s \gamma^\nu \frac{\lambda_{jl}^a}{2} v(p_2, \sigma_2)\right] \label{eq: amplitud}\\
    &= -\frac{g_s^2}{k^2} \frac{\lambda_{ki}^a}{2} \frac{\lambda_{jl}^a}{2}
    \overline{u}(q_1, \tau_1)\gamma^\mu u(p_1, \sigma_1)\overline{v}(q_2, \tau_2)\gamma^\nu v(p_2, \sigma_2)\nonumber
\end{align}
Definiendo el parámetro \(\kappa\):

\begin{align}
    k^2 = (p_1 - q_1)^2 = (E_{p_1} - E_{q_1})^2 - \kappa^2 \label{eq: kappa}
\end{align}
Agregando las contribuciones del color para las funciones de onda de los mesones.

\begin{align}
   \mathcal{M}_{gluon} &= -\frac{g_s^2}{\kappa^2} \frac{\lambda_{ki}^a}{2} \frac{\lambda_{jl}^a}{2}
    \frac{1}{\sqrt{3}}\delta_{ij}\frac{1}{\sqrt{3}}\delta_{kl}
    \overline{u}(q_1, \tau_1)\gamma^\mu u(p_1, \sigma_1)\overline{v}(q_2, \tau_2)\gamma^\nu v(p_2, \sigma_2)
\end{align}

Donde clarificando la notación de 
Einstein y usando las propiedades de las
matrices de Gell-mann

\begin{align*}
    &\sum_{a=1}^8\frac{\lambda_{ki}^a}{2} \frac{\lambda_{jl}^a}{2}\quad\cdot\quad
    \sum_{i,j=1}^3\frac{1}{\sqrt{3}}\delta_{ij}\sum_{k,l=1}^3\frac{1}{\sqrt{3}}\delta_{kl}\\
    &= \frac{1}{12} \sum_{a=1}^8\sum_{i,k=1}^3 \lambda_{ki}^a \lambda_{ik}^a\\
    &=\frac{1}{12}\sum_{a=1}^8 Tr[(\lambda^a)^2]\\
    &= \frac{4}{3}
\end{align*}

Por lo que la amplitud se puede escribir:
\begin{align*}
    \mathcal{M}_{gluon} = -\frac{4}{3}\frac{g_s^2}{k^2}
     \overline{u}(q_1, \tau_1)\gamma^\mu u(p_1, \sigma_1)\overline{v}(q_2, \tau_2)\gamma^\nu v(p_2, \sigma_2)
\end{align*}

Que tomando el límite no relativista (\(p \to 0\))
 se usa la normalización encontrada \ref{eq: norm} y 
 tomando \(\kappa\) en cambio de k \ref{eq: kappa}, se obtiene la expresión

\begin{align}
    \boxed{\mathcal{M}_{gluon} = -\frac{4}{3}\frac{g_s^2}{\kappa^2}
     \delta_{\tau_1, \sigma_1}\delta_{\tau_2, \sigma_2}}
\end{align}

Encontrando la contribución dada por el proceso de quark-antiquark \ref{fig: gluónico}
%%%%%%%%%%%%%%%%%%%%%%%%%%%%%%%%%%%%%%%%%%%%%%%%%%%%%%%%%%
\subsection{Aniquilación par}
El segundo diagrama de Feynman a estudiar denota la contribución en el canal 
t del proceso quark-antiquark, también conocido como proceso de aniquilación par.
\begingroup
\begin{figure}[H]
\begin{center}
\feynmandiagram [scale=2,transform shape][vertical'=a to b] {
i1 [particle=\(\overline{q}_l\)]
-- [fermion] a
-- [fermion] f1 [particle=\(q_k\)],
b -- [gluon, momentum'=P] a,
i2 [particle=\(q_i\)]
-- [fermion] b
-- [fermion] f2 [particle=\(\overline{q}_j\)],
};
\end{center}
\caption{Aniquilación par quark anti-quark.}
\label{fig: aniquilacion}
\end{figure}
\endgroup

Usando las leyes de Feynman sobre el diagrama \ref{fig: aniquilacion} se encuentra

\begin{align*}
    \mathcal{M}_{ani} = \frac{g_s^2}{P^2}\frac{1}{\sqrt{3}}\delta_{ij}
    \frac{1}{\sqrt{3}}\delta_{kl}\frac{\lambda_{ij}^a}{2}\frac{\lambda_{kl}^a}{2}
    \overline{u}(q_1, \tau_1)\gamma^\nu v(q_2, \tau_2)\overline{v}(q_2, \tau_2)\gamma^\mu u(p_1, \sigma_1)
\end{align*}

Analizando nuevamente las contribuciones del color
\begin{align}
    \boxed{\frac{1}{\sqrt{3}}\sum_{ij=1}^3 \delta_{ij}\frac{\lambda_{ij}^a}{2}
    = \frac{1}{2\sqrt{3}}Tr[\lambda^a] = 0 }
\end{align}

Por lo que la contribución de este diagrama se anula, dejando únicamente 
la amplitud para el intercambio gluónico.

\subsection{Deducción del potencial de Cornell}
La amplitud para el intercambio gluónico.

\begin{align}
    T_{fi} = \frac{1}{(2\pi)^6}\mathcal{M}_{gluon} = -\frac{1}{(2\pi)^6}
    \frac{4}{3}\frac{g_s^2}{\kappa^2} \delta_{\tau_1, \sigma_1}\delta_{\tau_2, \sigma_2}
\end{align}

Conociendo la amplitud de Scattering, es necesario tomar la transformada de Fourier

\begin{align}
    V_{Col}(x) &= (2\pi)^3 \int_{-\infty}^{\infty} d^3\kappa e^{-i \mathbf{\kappa \cdot x}}T_{fi}(\kappa)\nonumber\\
    &= - \frac{4}{3} \frac{g_s^2}{(2\pi)^3}\int_{-\infty}^{\infty} d^3\kappa e^{-i \mathbf{\kappa \cdot x}}\frac{1}{\kappa^2}
\end{align}

Sea el Ansatz

\begin{align*}
    \frac{1}{(2\pi)^3}\int_{-\infty}^{\infty} d^3\kappa e^{-i \mathbf{\kappa \cdot x}}\frac{1}{\kappa^2} = \frac{A}{r}
\end{align*}

Tomando Laplaciano a ambas partes, por la sección izquierda
\begin{align*}
    \frac{1}{(2\pi)^3}\mathbf{\nabla^2}
\int_{-\infty}^{\infty} d^3\kappa e^{-i \mathbf{\kappa \cdot x}}\frac{1}{\kappa^2}
= -\frac{1}{(2\pi)^3}
\int_{-\infty}^{\infty} d^3\kappa e^{-i \mathbf{\kappa \cdot x}} = - \delta^3(x)
\end{align*}

En la sección derecha
\begin{align*}
    \mathbf{\nabla^2} \frac{A}{r} = -4\pi A \delta^3(x)
\end{align*}
Por lo tanto, el factor A debe ser proporcional a

\begin{align}
    A = \frac{1}{4\pi}
\end{align}

Concluyendo que el potencial en su parte perturbativa toma la forma:

\begin{align}
    \boxed{
    V_{Col}(r) = -\frac{4}{3}\frac{g_s^2}{4\pi r} = -\frac{4}{3}\frac{\alpha_s}{r}
    }
\end{align}

Donde \(\alpha_s\) es la constante de interacción fuerte. Este potencial tipo Coulomb contribuye a distancias cortas, para considerar grandes distancias donde
el confinamiento es clave se supone,
\begin{align*}
    V_{conf}(r) = \sigma r^n \qquad n>0.
\end{align*}

Teniendo presente el espectro mesónico se deduce que \(n \approx 1\), por lo tanto
\begin{align}
    V_{conf}(r) = \sigma r ,
\end{align}

Finalmente se encuentra el potencial de Cornell \cite{cornell},

\begin{align}
    \boxed{V_{Cornell}(r) = -\frac{4}{3}\frac{\alpha_s}{r} + \sigma r}
\end{align}

El potencial depende directamente de los parámetros 
\(\alpha_s\) y \(\sigma\), que determinan el espectro quarkonico, mostrando un comportamiento como se observa ne la figura\ref{fig: PotCorn}
\begin{figure}[H]
    \centering
    \includegraphics[scale=0.75]{Imagenes/potencial.pdf}
    \caption{Potencial de Cornell}
    \label{fig: PotCorn}
\end{figure}

La anterior gráfica describe los términos más importantes dependiendo
los valores de r, para r pequeños el potencial tipo Coulomb es dominante, 
mientras que para r grandes lo es el potencial lineal de confinamiento.
