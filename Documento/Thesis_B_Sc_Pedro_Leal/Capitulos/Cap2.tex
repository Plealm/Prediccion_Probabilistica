\chapter{Ecuación de Schrödinger} \label{cap:2}
Con el potencial de Cornell definido en el anterior capítulo, se puede definir adecuadamente la ecuación de Schrödinger a solucionar, tal que, en este capítulo se solucionará por límites la formulación del problema, encontrando expresiones para la energía y función de onda con \(l = 0\).
\section{Solución de la ecuación de Schrödinger}

La ecuación de Schrödinger reemplazando el potencial genérico\(V_0\) \ref{eq: schro} con el potencial de Cornell, se obtiene la siguiente ecuación diferencial
\begin{align}
      \left[- \frac{\nabla^2}{m_Q} + V_{\text{Cornell}}\right]\psi(r, \theta, \phi) = E\psi(r, \theta, \phi)
\end{align}

En coordenadas esféricas se puede reescribir como,
\begin{align*}
  -\frac{1}{m_Q}\left[\frac{1}{r^2}\frac{\partial}{\partial r}
  \left(r^2 \frac{\partial\psi}{\partial r}\right) +
  \frac{1}{r^2 sin \theta} \frac{\partial}{\partial \theta}
  \left(sin \theta \frac{\partial\psi}{\partial \theta}\right)
  + \frac{1}{r^2 sin^2 \theta}\frac{\partial^2\psi}{\partial \phi^2}\right]
  + V_{\text{Cornell}}\psi = E \psi
\end{align*}

Como es un potencial central, se pueden separar las contribuciones radiales de las angulares, siendo posible usar separación de variables

\begin{align*}
  \psi(r, \theta, \psi) = R_{nl}(r)Y_{lm}(\theta, \phi)
\end{align*}

La solución de la parte angular es conocida, pero para la ecuación radial se usa la sustitución \(u=rR\), donde u se llama la función radial reducida, encontrando la misma forma de la ecuación de Schrödinger
\begin{align}
    -\frac{1}{m_Q}\frac{d^2u}{dr^2} + u\left[ V_{\text{Cornell}} + \frac{l(l+1)}{m_Qr^2}\right] =  u E
\end{align}
La cual igualando a 0,
\begin{align}
    \frac{d^2u}{dr^2} + u \left[E + \frac{4\alpha_s}{3r} - \sigma r - \frac{l(l+1)}{m_Q r^2}\right] m_Q = 0 \label{eq: EDO}
\end{align}

Esta última expresión \ref{eq: EDO}, permite denotar la dependencia de cada término con respecto a la variable r.
\subsection{$r \to 0$}
Suponiendo que r es muy pequeño se observa que los términos que más contribuyen son la parte de Coulomb del potencial y el factor con l, encontrando la ecuación diferencial
\begin{align}
      \frac{d^2 u}{dr^2} - u\frac{l(l+1)}{m_Q r^2}m_Q = 0
\end{align}
Esta formulación posee una solución conocida,

\begin{align}
   u(r \to 0) = r^{l+1}\label{eq: solradicero}
\end{align}
\subsection{$r \to \infty$}
Por otra parte, tomando r muy grandes, los términos más significativos son la parte lineal del potencial y  la energía.
\begin{align}
    \frac{d^2 u }{dr^2} = u[\sigma r - E]m_Q \label{eq: radialinf}
\end{align}
Esta ecuación diferencial se puede reducir tomando la sustitución \( w = \frac{(\sigma r - E)m_Q}{(m_Q \sigma)^{2/3}}\),
\begin{align*}
    \frac{d^2 u}{dw^2} &= u w
\end{align*}
Teniendo presente la definición de la función de Airy \(Ai(x)\) es la solución
de la ecuación diferencial

\begin{align*}
  \frac{d^2 y}{dx^2} = xy \qquad y = A\cdot Ai(x)+ B\cdot Bi(x)
\end{align*}
Esta ecuación posee dos soluciones linealmente independientes, de las cuales \(Ai(x)\)
es solución para \(y \to 0\) y \(x \to \infty\), por lo que solo se tomará en cuenta
esta solución. Entonces,

\begin{align}
  u(r \to \infty) \propto Ai\left(\frac{(\sigma r - E)m_Q}{(m_Q \sigma)^{2/3}}\right)\label{eq: solradiinf}
\end{align}

Esta deducción se aplica únicamente cuando \(l = 0\), por medio de la anterior
expresión se cuantizar la energía del sistema.

\subsection{Energía}
Despejando la energía de la solución asintótica a infinito de la función de onda \ref{eq: solradiinf}, se obtiene  que es \cite{base},

\begin{align}
    E = -\left(\frac{\sigma^2}{m_Q}\right)^\frac{1}{3}a_n \qquad (l = 0)
\end{align}

Donde \(a_n\) es el n-ésimo cero de la función de Airy (\(Ai(a_n) = 0\)). Es importante notar como esta energía es correcta cuando (l = 0), por lo que no se pueden describir
estados con momento angular orbital superiores. 

\subsection{Función de onda}
Teniendo presente las soluciones \ref{eq: solradiinf} y \ref{eq: solradicero},
se encuentra una función de onda,

\begin{align}
    u_{nl}= r^{l + 1}Ai(a_n + r(m_Q\sigma)^\frac{1}{3})\label{eq: solradial}
\end{align}

Para el caso \(l = 0\), la contribución dada por \ref{eq: solradicero} (\(r^{l + 1}\))
no es predominante, siendo la función de Airy la que determina principalmente la función de onda, por lo que se puede expresar,

\begin{align*}
    u_{nl}= Ai(a_n + r(m_Q\sigma)^\frac{1}{3}) \qquad (l = 0)
\end{align*}


\section{Minimización de $\chi^2$ para estados S}
Tomando la definición de la energía descrita anteriormente se define la masa como,
\begin{align}
    M = 2m_Q + E = 2m_Q -\left(\frac{\sigma^2}{m_Q}\right)^\frac{1}{3}a_n \label{eq: masa}
\end{align}

Se define la función \(\chi^2\), tal que,

\begin{align}
    \chi^2 = \sum_i^n \left(\frac{(M_{exp}^i - M_{teo}^i)^2}{\Delta M^i}\right)
\end{align}

Donde \(M_{exp}^i\) es el valor experimental del estado espectroscopio i,
\(\Delta M^i\) la incertidumbre experimental del estado i y \(M_{teo}^i\) en este caso es
el valor del estado con la masa encontrada parametrizada para el estado i 
\ref{eq: masa}.  Como la masa del estado depende únicamente de la masa del quark 
\(m_q\) y el número cuántico n, serán estos lo parámetros que se miminizarán, dado
que \(\alpha_s\) no se toma en cuenta en ninguna de las soluciones asintóticas y por tanto, no se encuentra una contribución directa a la energía.



\subsection{Charmonium}
Se tomaron los valores experimentales de los estados 1S, 2S del
PDG \cite{PDG}, para los estados 3S y 4S se tomaron los datos 
de \cite{data}. 


\begin{table}[H]
\resizebox{17cm}{!}{
\begin{tabular}{@{}cccccccc@{}}
\toprule
\textbf{} & Este trabajo[Gev] & Promedio[Gev]& BSE-CIA\cite{BSE-CIA} [Gev]& Expt  \cite{PDG} [Gev]& Pot. Model\cite{Pot}[Gev] & Lattice QCD \cite{lattice}[Gev]& \cite{data} [Gev] \\ \midrule
$M_{\eta_c}(1s)$     & 3.04$\pm$0.16 &3.0404$\pm$0.0002& 2.9509  & 2.9839$\pm$0.0004 & 2.980  & 3.292       & 2.981 \\
$M_{\eta_c}(2s)$     & 3.62$\pm$0.28 &3.6618$\pm$0.0006& 3.7352  & 3.6375$\pm$0.0011 & 3.600 & 4.240       & 3.635 \\
$M_{\eta_c}(3s)$     & 4.1$\pm$0.4   &4.0140$\pm$0.0007& 4.4486  &               & 4.060     &             & 3.989 \\
$M_{\eta_c}(4s)$     & 4.5$\pm$0.5   &4.4110$\pm$0.0005& 5.1048  &               & 4.554     &             & 4.401 \\ \bottomrule
\end{tabular}
\caption{Resultados de las masa del charmonium para los estados S comparados con diferentes valores teóricos y experimentales.}
\label{table: 1}
}
\end{table}

Los resultados presentados en la tabla \ref{table: 1} fueron 
obtenidos realizando la minimización mostrada en la figura \ref{fig: chi1}, donde el mínimo de \(\chi^2 =  7.3343\)
valor que es mucho mayor a uno, esto es explicable por 
la baja incertidumbre que poseen los valores experimentales, por lo tanto,
para asegurar el cálculo de las incertidumbres se reescalan las mismas
por la raíz cuadrada de este mínimo.

\begin{figure}[H]
    \centering
    \includegraphics[scale=0.9]{Imagenes/chi2Charmonium.pdf}
    \caption{Minimización del \(\chi^2\) del Charmonium para el 
    estado S.}
    \label{fig: chi1}
\end{figure}

y encontrando unos valores de los parámetros 
\begin{equation}
\boxed{
\begin{array}{rcl}
 \sigma &= \SI{0.201\pm0.062}{\giga\electronvolt\squared\per\femto\meter}\\
    m_c &= \SI{1.138\pm0.011}{\giga\electronvolt}
\end{array}
}
\end{equation}

La masa del quark charm es bastante cercano al valor dado por el PDG\cite{PDG} \(\SI{1.27\pm 0.02}{\giga\electronvolt}\). Existen diferentes formas de calcular
la constante de acople \(\alpha_s\), entre ellas está la libertad asintótica en 'naive dimensional regularization'\cite{Maltoni} a NLO,

\begin{align}
    \alpha_s(\mu) = \frac{4\pi}{\beta_0 ln\left(\frac{\mu^2}{\Lambda_{QCD}^2}\right)}\left[1 - \frac{\beta_1 ln\left[ln\left(\frac{\mu^2}{\Lambda_{QCD}^2}\right)\right]}{\beta_0^2ln\left(\frac{\mu^2}{\Lambda_{QCD}^2}\right)}\right] \label{eq: libertad}
\end{align}

Donde, 
\begin{align*}
    \beta_0 = 11 - \frac{2}{3}n_f \qquad \beta_1 102 - \frac{38}{3}nf
\end{align*}

Con \(n_f=5\) el número de colores en el estado singlete.Por medio de la expresión
\ref{eq: libertad} se puede encontrar \(\alpha_s\) dependiendo de la masa del quark,  para el charmonium,

\begin{align}
    \boxed{\alpha_s(m_c) = 0.4011(23)}
\end{align}

\subsection{Bottomonium}
De igual manera los primeros estados fueron tomados los datos del Particle data group \cite{PDG} y los estados 3S y 4S de \cite{data},


\begin{table}[H]
\resizebox{17cm}{!}{
\begin{tabular}{@{}cccccccc@{}}
\toprule
\textbf{} & Este trabajo[GeV]  & Promedio[GeV]  & BSE-CIA\cite{BSE-CIA}[GeV] & Expt  \cite{PDG} [GeV]& Pot. Model\cite{Pot} [GeV]& Lattice QCD \cite{lattice}[GeV]& \cite{data} [GeV] \\ \midrule
$M_{\eta_b}(1s)$     & 9.450$\pm$0.030  &9.4295$\pm$0.0010          &9.0005    & 9.3987$\pm$0.002 & 9.390    &  7.377       &  9.398 \\
$M_{\eta_b}(2s)$     & 9.96$\pm$0.04    &10.0111$\pm$0.0020         &9.7215    & 9.999$\pm$0.004 & 9.947      &  8.202       & 9.990 \\
$M_{\eta_b}(3s)$     & 10.38$\pm$0.06   &10.3421$\pm$0.0006         & 10.4201  &               &  10.291  &             &  10.329 \\
$M_{\eta_b}(4s)$     & 10.74$\pm$0.07   &10.576$\pm$0.006         & 11.0968  &               &          &             & 10.573 \\ \bottomrule
\end{tabular}
\caption{Resultados de las masa del Bottomonium para los estados S comparados con diferentes valores teóricos y experimentales.}
\label{table: 2}
}
\end{table}

Por medio de la minimización mostrada en la figura \ref{fig: chi2}
se encontraron los valores de \ref{table: 2}, donde se refleja que 
el mínimo de la función es de \(\chi^2 = 7.21393\), este valor es usado 
como se explicó anteriormente para el cálculo de incertidumbres.

\begin{figure}[H]
    \centering
    \includegraphics[scale=0.9]{Imagenes/chi2Bottonium.pdf}
    \caption{Minimización del \(\chi^2\) del Bottomonium para el 
    estado S.}
    \label{fig: chi2}
\end{figure}

y encontrando unos valores de los parámetros 
\begin{equation}
\boxed{
\begin{array}{rcl}
 \sigma &= \SI{0.329\pm0.016}{\giga\electronvolt\squared\per\femto\meter}\\
    m_b &= \SI{4.385\pm0.011}{\giga\electronvolt}
\end{array}
}
\end{equation}

La masa del quark bottom dado por PDG se encuentra dentro de la incertidumbre \(\SI{4.18\pm0.04}{\giga\electronvolt}\)\cite{PDG}. Análogamente, usando la libertad asintótica para determinar la constante de interacción de la QCD para la masa
del Bottom, se obtiene,

\begin{align}
    \boxed{\alpha_s(m_b) = 0.22611(17)}
\end{align}

Que al compararse con el valor encontrado con el Charm, se encuentra el comportamiento
esperado, donde a mayor masa del quark menor será \(\alpha_s\) para que se cumpla 
la libertad asintótica.