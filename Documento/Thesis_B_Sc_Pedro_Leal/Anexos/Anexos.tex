\begin{appendix}

\chapter{Reglas de Feynman para la QCD}\label{AnexoA}

En el presente apéndice se presentan las reglas de Feynamn 
más importantes para el análisis de interacciones de la QCD
para un quark y antiquark\cite{libroqcd}.
\begin{table}[H]
\resizebox{13cm}{!}{
\begin{tabular}{|c|c|c|}
\hline
Quark Entrante      &  
\feynmandiagram [horizontal=a to b]
{a -- [fermion, edge label=\(p\)] b[dot]}; 
&$u(p)$
\\\hline
Quark Saliente      &  
\feynmandiagram [horizontal=a to b]
{b -- [fermion, edge label'=\(p\)] a[dot]}; 
& $\overline{u}(p)$ 
\\\hline
Anti-Quark Entrante & 
\feynmandiagram [horizontal=a to b]
{a -- [fermion, edge label=\(p\)] b[dot]}; 
&$v(p)$\\ \hline
Anti-Quark Saliente &  
\feynmandiagram [horizontal=a to b]
{b -- [fermion, edge label'=\(p\)] a[dot]}; 
& $\overline{v}(p)$ \\ \hline
\begin{tabular}{c}
Propagador gluónico
\end{tabular} &  
\begin{tabular}{c}
\feynmandiagram [horizontal=a to b]
{a[particle={\(a, \mu\)}] -- [gluon, momentum=\(q\)] b[particle={\(b, \nu\)}]}; 
\end{tabular}& $-i\frac{g_{\mu \nu}\delta^{a, b}}{q^2}$ \\ \hline
Vértice gluon-quark & 
\begin{tabular}{c}
 \feynmandiagram [horizontal=a to b] {
a[particle={\(a, \mu\)}] -- [gluon] b,
f1[particle={\(i\)}] -- [fermion] b -- [fermion] f2[particle={\(j\)}],
};
\end{tabular}
&$-i\frac{g_s}{2} \lambda_{ij}^a\gamma^\mu$\\ \hline
\end{tabular}
}
\caption{Reglas de Feynman para procesos base de QCD.}
\end{table}

Tal que, la contribución del color se encuentra en los índices de las matrices de 
Gell-Mann, por ejemplo\cite{dutch}:

\begin{align*}
    c_i^\dagger\lambda^ac_j = \langle i|\lambda|j\rangle
\end{align*}

Para calcular estos índices en un nivel árbol, es preciso recordar

\begin{align*}
    \frac{1}{4}\sum_a\lambda^a_{ij}\lambda^a_{kl}= \frac{1}{2}\delta_{il}\delta_{kj}
    - \frac{1}{6}\delta_{ij}\delta_{kl}\equiv f(ijkl)
\end{align*}
Donde,
\begin{align}
\centering
f(ijkl)= \{ \begin{array}{lc}
              -\frac{1}{6 }&  \text{si i = j } \text{y k = l} \\\\
              \frac{1}{2} &  \text{si i = l } \text{y j = k} \\\\
              \frac{1}{3}  &\text{si i = j = k = l} \\\\
              0 &  \text{en otro caso}
              \end{array}
   . 
\end{align}

%%%%%%%%%%%%%%%%%%%%%%%%%%%%%%%%%%%%%%%%%%%%%%%%%%%%%%%%%%%%%%
\chapter{Deducción de las correcciones Espinoriales}\label{AnexoCorrec} 
Los cálculos presentados en este apéndice fueron aprendidos y realizados siguientes el paper de lucha(1995) \cite{hiperfina}.

Retomando los diagramas de Feynman de intercambio gluónico \ref{fig: gluónico} y la aniquilación \ref{fig: aniquilacion} presentados en el capítulo \ref{cap:1} se puede ampliar las contribuciones si se toma en cuenta un nuevo proceso, la interacción de fermion y anti-fermion\cite{hiperfina},
\begin{align}
    \mathcal{F}(p_1, \sigma_1)\overline{\mathcal{F}}(p_2, \sigma_2) \to 
    \mathcal{F}(q_1, \tau_1) \overline{\mathcal{F}}(q_2, \tau_2)
\end{align}

transformando la expresión \ref{eq: operadores}, 
\begin{align*}
    u(p, \sigma) = 
    \begin{pmatrix}
    1 \\ \frac{\mathbf{\sigma \cdot p}}{E + m}
    \end{pmatrix} \chi_\sigma, \\
    v(p, \sigma) = 
    \begin{pmatrix}
    \frac{\mathbf{\sigma \cdot p}}{E + m}\\ 1
    \end{pmatrix} \chi_\sigma^c,
\end{align*}


Este proceso permite encontrar nuevas contribuciones, pero por el momento solo se ha analizado la parte vectorial, para analizar las otras 
estructuras de Lorentz es necesario generalizar la amplitud encontrada en \ref{eq: amplitud}, definiendo dos espinores bilineales de Dirac,

\begin{align*}
    &\overline{u}(q_1, \tau_1)\Gamma_1u(p_1, \sigma_1)\\
    &\overline{v}(p_2, \sigma_2)\Gamma_2v(q_2, \tau_2)
\end{align*}

Tal que, \(\Gamma_1\) y \(\Gamma_2\) están definidas por medio de matrices
de Dirac. Un kernel de interacción \(\mathcal{K}\) que depende de los momentos,

\begin{align*}
    \mathcal{K} = \mathcal{K}(p_1, p_2, q_1, q_2)
\end{align*}

Por lo anterior, se puede generalizar la amplitud,

\begin{align}
    i\mathcal{M} = \overline{u}(q_1, \tau_1)\Gamma_1u(p_1, \sigma_1)
    \overline{v}(p_2, \sigma_2)\Gamma_2v(q_2, \tau_2)\mathcal{K}
\end{align}

Reemplazando \(\Gamma_1\) y \(\Gamma_2\) con las diferentes estructuras de Lorentz y realizando todo el proceso analítico se encuentra,

\begin{table}[H]
\begin{tabular}{ccc}
\hline
Estructura de Lorentz & \(\Gamma_E\bigotimes\Gamma_E\)   & Potencial estático    \\ \hline
Escalar               & 1$\bigotimes$1        & $V_s(r)$         \\
Pseudo-escalar        & $\gamma_5\bigotimes\gamma^5$     & 0             \\
vectorial             & $\gamma_\mu\bigotimes\gamma^\mu$   & $V_v(r)$         \\
vectorial axial       & $\gamma_\mu\gamma_5\bigotimes\gamma^\mu\gamma^5$ & $4 S_1S_2V_A(r)$ \\
Tensor                & $\frac{1}{2}\sigma_{\mu, \nu}\bigotimes\sigma^{\mu, \nu}$     & $4 S_1S_2V_T(r)$ \\ \hline
\end{tabular}
\caption{Diferentes contribuciones del potencial desde las estructuras 
de Lorentz de una interacción genérica fermion-antifermion\cite{hiperfina}.}
\end{table}

Donde, 
\begin{align*}
    \sigma_{\mu, \nu} = \frac{i}{2}[\gamma_\mu,\gamma_\nu]
\end{align*}

y el operador, 
\begin{align*}
    \gamma^5 = -\begin{pmatrix}
    0 & 1\\ 1 & 0
    \end{pmatrix}
\end{align*}
Tal que, \(V_s(r)\) es el potencial escalar encontrado de la transformada de Fourier para el caso escalar, análogamente los demás. Los únicos potenciales puros se dan en las estructuras vectorial y escalar,

\section{Estructura Vectorial}
Tomando la estructura vectorial se observa,

\begin{align}
         i\mathcal{M} &= \overline{u}(q_1)\gamma_\mu u(p_1)
    \overline{v}(p_2)\gamma^\mu v(q_2)\mathcal{K}_v\nonumber\\
    &=\overline{u}(q_1) u(p_1)\overline{v}(p_2)v(q_2) - \overline{u}(q_1)\gamma_0\mathbf{\gamma} u(p_1)
    \overline{v}(p_2)\gamma_0\mathbf{\gamma} v(q_2)\nonumber\\
    &= \left[\chi^\dagger\begin{pmatrix}
    1, \frac{\sigma\cdot q_1}{2m}
    \end{pmatrix} \begin{pmatrix}
    1 \\ \frac{\sigma\cdot p_1}{2m}
    \end{pmatrix}\chi\chi^{c\dagger}\begin{pmatrix}
    \frac{\sigma\cdot p_2}{2m}, 1
    \end{pmatrix} \begin{pmatrix}
    \frac{\sigma\cdot q_2}{2m}\\1
    \end{pmatrix}\chi^c\right]\nonumber\\&-\left[\chi^\dagger\begin{pmatrix}
    1, \frac{\sigma\cdot q_1}{2m}
    \end{pmatrix}\begin{pmatrix}
    0 & \sigma\\
    \sigma & 0
    \end{pmatrix}\begin{pmatrix}
    1 \\\frac{\sigma\cdot p_1}{2m}
    \end{pmatrix}\chi
    \chi^{c\dagger}\begin{pmatrix}
    \frac{\sigma\cdot p_2}{2m}, 1
    \end{pmatrix}\begin{pmatrix}
    0 & \sigma\\
    \sigma & 0
    \end{pmatrix}\begin{pmatrix}
    \frac{\sigma\cdot p_1}{2m}\\1
    \end{pmatrix}\chi^c\right]
\end{align}
Realizando las operaciones matriciales y usando la normalización para caso no relativista,

\begin{align}
    &= \left[1 + \frac{1}{16m^2}(\chi^\dagger(\sigma\cdot q_1)(\sigma\cdot p_1)\chi\chi^{c\dagger}(\sigma\cdot p_2)(\sigma\cdot q_2)\chi^c)+ 
    \frac{1}{4m^2}\chi^{c\dagger}(\sigma\cdot p_2)(\sigma\cdot q_2)\chi^c
    + \frac{1}{4m^2}\chi^{\dagger}(\sigma\cdot q_1)(\sigma\cdot p_1)\chi\right]\nonumber\\
    &- \frac{1}{4m^2}[\chi^\dagger\sigma (\sigma\cdot p_1)\chi\chi^{c\dagger}(\sigma\cdot p_2)\sigma\chi^c + \chi^\dagger\sigma (\sigma\cdot p_1)\chi\chi^{c\dagger}\sigma(\sigma\cdot q_2)\chi^c 
    +\chi^\dagger(\sigma\cdot q_1)\sigma\chi\chi^{c\dagger}(\sigma\cdot p_2)\sigma\chi^c 
    \nonumber\\ &+\chi^\dagger(\sigma\cdot q_1)\sigma\chi\chi^{c\dagger}\sigma(\sigma\cdot q_2)\chi^c]
\end{align}

Despreciando los términos poco contributivos y reescribiendo,

\begin{align}
    &= 1 + \frac{1}{4m^2}[\chi^{c\dagger}(\sigma\cdot q_2)(\sigma\cdot p_2)\chi^c + \chi^\dagger (\sigma\cdot q_1)(\sigma\cdot p_1)\chi + 
    \left[\chi^\dagger( (\sigma\cdot q_1)\sigma + \sigma(\sigma\cdot p_1))\chi\right]\nonumber\\&\left[\chi^{c\dagger} (\sigma(\sigma\cdot p_2) + (\sigma\cdot q_2))\chi^c\right]]
\end{align}

Teniendo presente,

\begin{align*}
    \chi^{c\dagger}(\sigma\cdot q_2)(\sigma\cdot p_2)\chi^c  = \chi^{c\dagger}P_iK_ji \epsilon_{ijk}\sigma_k\chi^c = i\epsilon_{ijk}P_iK_j\sigma_{1k}\\
    \chi^\dagger (\sigma\cdot q_1)(\sigma\cdot p_1)\chi = \chi^{\dagger}P_iK_ji \epsilon_{ijk}\sigma_k\chi = i\epsilon_{ijk}P_iK_j\epsilon_{ijk}\sigma_{2k}\\
    \chi^\dagger( (\sigma\cdot q_1)\sigma + \sigma(\sigma\cdot p_1))\chi = 
    \chi^\dagger [2(P_j - K_j)+ i \epsilon_{ijk}K_j\sigma_{1k}]\chi\\
    \chi^{c\dagger} (\sigma(\sigma\cdot p_2) + (\sigma\cdot q_2))\chi^c = 
    \chi^{c\dagger} [2(P_j - K_j)+ i \epsilon_{ijk}K_j\sigma_{2k}]\chi^c
\end{align*}

Lo que permite escribir

\begin{align}
    \boxed{\mathcal{M} = [1 + \frac{1}{4m^2}(3i\epsilon_{ijk}(\sigma_{1k} + \sigma_{2k}) - (\mathbf{\sigma_1}\cdot\mathbf{\sigma_2})\mathbf{k}^2 + (\mathbf{\sigma_1}\cdot \mathbf{k})(\mathbf{\sigma_2}\cdot \mathbf{k}))]\mathcal{K}_v}
\end{align}

Tomando la transformada de Fourier para las diferentes dependencias 
de k,

\begin{align*}
    &\frac{1}{(2\pi)^3}\int d^3k k_j e^{-i \mathbf{k}\cdot \mathbf{x}} \mathcal{K}_v = - i \frac{x_j}{r} V'_v(r) \qquad \text{orbita-spin}
\end{align*}

\begin{align*}
    &\frac{1}{(2\pi)^3}\int d^3k \mathbf{k}^2 e^{-i \mathbf{k}\cdot \mathbf{x}} \mathcal{K}_v = \triangle V_v(r) \qquad \text{spin-spin}
\end{align*}

\begin{align*}
    &\frac{1}{(2\pi)^3}\int d^3k k_i k_j e^{-i \mathbf{k}\cdot \mathbf{x}} \mathcal{K}_v =\nabla_i \nabla_j V_v(r)  \\ &= \left(\frac{x_ix_j}{r^2} - \frac{1}{3}\delta_{ij} \right)\left[V''_v(r) - \frac{1}{r}V'_v(r)\right] + \frac{1}{3}\delta_{ij} \triangle V_v(r)\qquad \text{Tensorial}
\end{align*}

Por lo tanto, el término Spin-Orbita

\begin{align*}
    H_{LS}^{Vectorial} = \frac{3}{2m^2r}(\mathbf{x} \times \mathbf{p})\cdot \mathbf{S}V'_v(r) = \frac{3}{2m^2r}\mathbf{L}\cdot \mathbf{S}V'_v(r)
\end{align*}

En el caso de la contribución Spin-Spin

\begin{align*}
    H_{SS}^{Vectorial} = \frac{2}{3m^2}\mathbf{S_1}\cdot\mathbf{S_2}\triangle V_v(r)
\end{align*}

Finalmente el término tensorial,

\begin{align*}
    H_{T}^{Vectorial} &= \frac{1}{m^2}\left[\frac{(\mathbf{S}_1 \cdot \mathbf{x})(\mathbf{S}_2 \cdot \mathbf{x})}{r^2} - \frac{1}{3}\mathbf{S}_1\cdot\mathbf{S}_2\right]\left[\frac{1}{r}V'_v(r)- V''_v(r)\right]\\
    &= \frac{1}{12m^2}S_{12}\left[\frac{1}{r}V'_v(r)- V''_v(r)\right]
\end{align*}

Conociendo que 

\begin{align}
    V_v(r) = - \frac{1}{(2\pi)^3}\int d^3ke^{-i\mathbf{k}\cdot\mathbf{x}}\mathcal{K}_v=
    - \frac{1}{(2\pi)^3}\int d^3ke^{-i\mathbf{k}\cdot\mathbf{x}}\left(-\frac{4g_s^2}{3\mathbf{k}^2}\right)
\end{align}

Se encuentra que las correcciones vectoriales son, 

\begin{align}
    V_{SS}^{Vectorial}(r) &= \frac{8 \alpha_s}{9m^2_Qr^2}(S_1 \cdot S_2)\delta(r)\nonumber\\
    V_{LS}(r)^{Vectorial} &= \frac{2 \alpha_s}{m^2_Qr^3}(L \cdot S) \\
    V_{T}(r)^{Vectorial} &=\frac{\alpha_s}{3m^2_Qr^2}S_{12}\nonumber
\end{align}
\section{Estructura Escalar}

De igual manera que anteriormente se toma la estructura,

\begin{align}
    i\mathcal{M} &= \overline{u}(q_1)u(p_1)
    \overline{v}(p_2)v(q_2)\mathcal{K}_s\nonumber\\
    &=\overline{u}(q_1)\gamma_0 u(p_1)\overline{v}(p_2)\gamma_0v(q_2)\nonumber\\
    & = - \left(1 - \frac{1}{4m^2}\left[\chi^\dagger (\sigma\cdot q_1)(\sigma\cdot p_1)\chi + \chi^{c\dagger} (\sigma\cdot p_2)(\sigma\cdot q_2)\chi^c\right]\right)\mathcal{K}_s
\end{align}
Usando las abreviaciones ubicando el sistema en su centro de masa y dejando de lado 
las contribuciones dependientes de spin que fueron usadas en el cálculo vectorial,
\begin{align}
    \mathcal{M} = - \left[1 - \frac{1}{4m^2}i\epsilon_{ijk}p_ik_j(\sigma_{1k} + \sigma_{1k})\right]\mathcal{K}_s
\end{align}

Realizando análogamente la transformada de Fourier para spin-orbita

\begin{align*}
    &\frac{1}{(2\pi)^3}\int d^3k k_j e^{-i \mathbf{k}\cdot \mathbf{x}} \mathcal{K}_s = i \frac{x_j}{r} V'_s(r) 
\end{align*}

Encontrando que la contribución escalar posee la forma,

\begin{align}
    H_{LS}^{Escalar} = -\frac{1}{2m^2r}\mathbf{L}\cdot\mathbf{S}V'_s(r)
\end{align}

Aunque se conoce la forma de \(V_s\), no se puede encontrar el valor exacto, dado
que no se conoce el valor de \(\mathcal{K}_s\), tal que,

\begin{align}
    V_s(r) = \frac{1}{(2\pi)^3}\int d^3k e^{-i \mathbf{k}\cdot \mathbf{x}}\mathcal{K}_s
\end{align}

Pero como corresponde a la forma funcional encontrada en \cite{base}, se toma

\begin{align}
    V_{LS}(r)^{Escalar} &= -\frac{\sigma}{2m^2_Qr}(L \cdot S)
\end{align}
%%%%%%%%%%%%%%%%%%%%%%%%%%%%%%%%%%%%%%%%%%%%%%%%%%%%%%%%%%%
\chapter{Integración de la función de Airy}\label{AnexoB}


Para agregar las contribuciones de las correcciones es necesario integrar la función
de onda al cuadrado se toma la forma genérica de la función para las correcciones
spin-orbita y spin-spin,

\begin{align*}
    \int_0^\infty r^m [Ai(a_n + r(m_Q\sigma)^\frac{1}{3})]^2dr
\end{align*}

Por facilidad de cálculo se tomará \(a = a_n\) y \(b = (m_Q\sigma)^\frac{1}{3}\).
Se toma de base la integral conocida y descrita 
en \cite{integral},

\begin{align}
    \int_0^\infty r^m [Ai(r)]^2 dr = J_m(2) = \frac{(\frac{2}{3})^\frac{2}{3}\Gamma(m + 1)}
    {4\sqrt{3\pi}\Gamma(\frac{m}{3} + \frac{7}{6})} = W_1(m)
\end{align}

Tomando la sustitución

\begin{align*}
    r&= a + zb &  r &arrow \infty &  z &arrow \infty\\
    dr&= b dz &  r &arrow 0 &  z &arrow \frac{-a}{b}
\end{align*}

Por lo que se reescribe como,

\begin{align*}
    b\int_\frac{-a}{b}^\infty (a + zb)^m [A_i(a + zb)]^2dz = W_1 &\\
    b(\int_\frac{-a}{b}^\infty (a + zb)^m [A_i(a + zb)]^2dz - \int_\frac{-a}{b}^0 (a + zb)^m [A_i(a + zb)]^2dz)= W_1 -  &b\int_\frac{-a}{b}^0 (a + zb)^m [A_i(a + zb)]^2dz\\
    b\int_0^\infty (a + zb)^m [A_i(a + zb)]^2dz= W_1 -  &b\int_\frac{-a}{b}^0 (a + zb)^m [A_i(a + zb)]^2dz
\end{align*}

Entonces, se realiza la integral desconocida
mediante el programa \textbf{Mathematica} encontrando.

\begin{align}
    b\int_\frac{-a}{b}^0 (a + zb)^m [A_i(a + zb)]^2dz = \frac{2^{-\frac{2 m}{3}-\frac{7}{3}} 3^{\frac{2 (m-1)}{3}}
   G_{2,4}^{3,1}((\frac{2}{3})^{2/3}
   a,\frac{1}{3}\Big|
\begin{array}{c}
 1,\frac{1}{6} (2 m+7) \\
 \frac{m+1}{3},\frac{m+2}{3},\frac{m+3}{3},0 \\
\end{array}
)}{\pi ^{3/2}} = W_2(m, a)
\end{align}
Integración que se cumple con la condición \(\Re(m) > -1\). G es la función de Meijer G, la cual es la generalización de gran cantidad de funciones especiales. Sustituyendo los resultados:
\begin{align}
    \int_0^\infty (a + zb)^m [A_i(a + zb)]^2dz&= \frac{W_1(m) - W_2(m, a)}{b} = W_3(m, a, b)
\end{align}

Usando el binomio de Newton

\begin{align*}
    \sum_{k=0}^m \begin{pmatrix}
    m \\ k
    \end{pmatrix}\int_0^\infty
    a^{m-k}z^k b^k[Ai(a + z b)]^2 dz &=  W_3(m, a, b)\\
    \sum_{k=0}^m\int_0^\infty
    z^k[Ai(a + z b)]^2 dz &= \sum_{k=0}^m\frac{W_3(m, a, b) k! (m - k)!}{m! a^{m-k}b^k}
\end{align*}
\begin{align}
    \boxed{
    \int_0^\infty
    z^m[Ai(a + z b)]^2 dz = \sum_{k=0}^m\frac{W_3(m, a, b) k! (m - k)!}{m! a^{m-k}b^k} - \sum_{k=0}^{m-1}\int_0^\infty
    z^k[Ai(a + z b)]^2 dz
    } \label{eq: integral1}
\end{align}

\section{Integración con gaussiana}

Como la corrección spin-spin depende de 
una delta de Dirac, esta se puede reemplazar como
\(\delta(r) = (\frac{Q}{\sqrt{\pi}})^3 e^{-Q^3r^2}\)\cite{delta}, para calcular esta contribución es necesario realizar la integral,

\begin{align}
    \int_0^\infty z^m[Ai(a + zb)]^2(\frac{Q}{\sqrt{\pi}})^3 e^{-Q^3z^2}dz 
\end{align}

la gaussiana se puede sustituir por la
expasión en serie de Taylor de la exponencial,
obteniendo,

\begin{align*}
    &= (\frac{Q}{\sqrt{\pi}})^3\int_0^\infty z^m[Ai(a + zb)]^2 
    \sum^\infty_{m' = 0} \frac{(- Q^3 z^2)^{m'}}{m'!}dz\\
    &= \sum^\infty_{m' = 0} (\frac{Q^{3(1+ m')}}{\pi^\frac{3}{2}m'!})
    \int_0^\infty z^{m+2m'}[Ai(a + zb)]^2dz
\end{align*}

Sea el resultado \ref{eq: integral1} igual a \(W_4(m,a,b)\), por lo anterior,

\begin{align}
    &= \sum^\infty_{m' = 0} (\frac{Q^{3(1+ m')}}{\pi^\frac{3}{2}m'!})W_4(m + 2m',a,b) \approx \boxed{\sum^5_{m' = 0} (\frac{Q^{3(1+ m')}}{\pi^\frac{3}{2}m'!})W_4(m + 2m',a,b)}
\end{align}

%%%%%%%%%%%%%%%%%%%%%%%%%%%%%%%%%%%%%%%%%%%%%%%%%%%%%%%%%%%
\chapter{Evaluación numérica de las correcciones}\label{Anexonum}

En el presente anexo se demuestran las formas de las contribuciones spinoriales
dependiendo únicamente de los parámetros \(m_Q, \sigma, \alpha\) y Q, esta último
apareciendo en las correcciones Spin-Spin,

\section{Estados S}
En los estados S solo se puede encontrar la corrección Spin-Spin, para los estados \(n ^1S_1(\eta(nS))\)
\begin{align}
    E_{\text{spin-spin}}^{1^1S_1} &=-\frac{1}{\text{mq}^{10}
   \sigma ^8 (\text{mq} \sigma )^{7/3}} (4.86\times10^{-22} \alpha  Q^3 (1.21\times 10^{20} \text{mq}^{10} \sigma ^{10}+6.15\times
   10^{15} \text{mq}^8 Q^3 \sigma ^8 \nonumber\\  &(-22322.7 \text{mq}^2 \sigma ^2+10745.6 \text{mq} \sigma +21535.2)-5.78217\times 10^{13}
   \text{mq}^6 Q^6 \sigma ^6 \nonumber\\ &(1.64\times 10^6 \text{mq}^4 \sigma ^4-1.10301\times 10^6 \text{mq}^3 \sigma ^3+1.28017\times 10^6
   \text{mq}^2 \sigma ^2-1.20737\times 10^6 \text{mq} \sigma\nonumber\\ & -2.38805\times 10^6)+1.30371\times 10^9 \text{mq}^4 Q^9 \sigma ^4
   (-2.67681\times 10^{10} \text{mq}^6 \sigma ^6+1.7811\times 10^{10} \text{mq}^5 \sigma ^5\nonumber\\ &-2.12754\times 10^{10} \text{mq}^4 \sigma
   ^4+2.59831\times 10^{10} \text{mq}^3 \sigma ^3-3.86115\times 10^{10} \text{mq}^2 \sigma ^2+4.5408\times 10^{10} \text{mq} \sigma\nonumber\\ 
   &+9.46711\times 10^{10})+26996.6 \text{mq}^2 Q^{12} \sigma ^2 (-3.37784\times 10^{14} \text{mq}^8 \sigma ^8+2.2086\times 10^{14}
   \text{mq}^7 \sigma ^7\nonumber\\ &-2.6238\times 10^{14} \text{mq}^6 \sigma ^6+3.23414\times 10^{14} \text{mq}^5 \sigma ^5-4.9628\times 10^{14} \text{mq}^4
   \sigma ^4+7.2159\times 10^{14} \text{mq}^3 \sigma ^3\nonumber\\ &-1.22789\times 10^{15} \text{mq}^2 \sigma ^2+1.58202\times 10^{15} \text{mq} \sigma
   +3.59258\times 10^{15})\nonumber\\ &+Q^{15} (-1.86673\times 10^{18} \text{mq}^{10} \sigma ^{10}+1.20543\times 10^{18} \text{mq}^9 \sigma
   ^9-1.42551\times 10^{18} \text{mq}^8 \sigma ^8\nonumber\\ &+1.75671\times 10^{18} \text{mq}^7 \sigma ^7-2.69936\times 10^{18} \text{mq}^6 \sigma
   ^6+3.954\times 10^{18} \text{mq}^5 \sigma ^5-6.89901\times 10^{18} \text{mq}^4 \sigma ^4\nonumber\\ &+1.07175\times 10^{19} \text{mq}^3 \sigma
   ^3-2.04905\times 10^{19} \text{mq}^2 \sigma ^2+2.60098\times 10^{19} \text{mq} \sigma +7.03226\times 10^{19})))
\end{align}

\begin{align}
    E_{\text{spin-spin}}^{2^1S_1} &=-\frac{1}{\text{mq}^{10}\sigma ^8 (\text{mq} \sigma )^{7/3}}
    (1.82185\times10^{-24} \alpha  Q^3 (4.2386\times 10^{22} \text{mq}^{10} \sigma ^{10}+5.36656\times
   10^{17} \text{mq}^8 Q^3 \sigma ^8\nonumber\\  &(-89512.5 \text{mq}^2 \sigma ^2+75337. \text{mq} \sigma +263977.)-2.88787\times 10^{15}
   \text{mq}^6 Q^6 \sigma ^6 \nonumber\\  &(1.14872\times 10^7 \text{mq}^4 \sigma ^4-1.35812\times 10^7 \text{mq}^3 \sigma ^3+2.7973\times 10^7
   \text{mq}^2 \sigma ^2-5.53639\times 10^7 \text{mq} \sigma \nonumber\\  &-2.14383\times 10^8)+1.21828\times 10^{10} \text{mq}^4 Q^9 \sigma ^4
   (-1.03781\times 10^{12} \text{mq}^6 \sigma ^6+1.20967\times 10^{12} \text{mq}^5 \sigma ^5\nonumber\\  &-2.58479\times 10^{12} \text{mq}^4 \sigma
   ^4+6.64088\times 10^{12} \text{mq}^3 \sigma ^3-1.88944\times 10^{13} \text{mq}^2 \sigma ^2+4.74264\times 10^{13} \text{mq} \sigma\nonumber\\  &
   +1.83196\times 10^{14})+82525.9 \text{mq}^2 Q^{12} \sigma ^2 (-4.13939\times 10^{16} \text{mq}^8 \sigma ^8+4.68053\times 10^{16}
   \text{mq}^7 \sigma ^7\nonumber\\  &-9.89769\times 10^{16} \text{mq}^6 \sigma ^6+2.56099\times 10^{17} \text{mq}^5 \sigma ^5-7.56813\times 10^{17}
   \text{mq}^4 \sigma ^4\nonumber\\  &+2.34627\times 10^{18} \text{mq}^3 \sigma ^3-7.39225\times 10^{18} \text{mq}^2 \sigma ^2+2.04527\times 10^{19} \text{mq}
   \sigma +7.8157\times 10^{19})+Q^{15} \nonumber\\  &(-7.18381\times 10^{20} \text{mq}^{10} \sigma ^{10}+7.90922\times 10^{20} \text{mq}^9 \sigma
   ^9-1.65545\times 10^{21} \text{mq}^8 \sigma ^8+4.27191\times 10^{21} \text{mq}^7 \sigma ^7\nonumber\\  &-1.26385\times 10^{22} \text{mq}^6 \sigma
   ^6+3.94727\times 10^{22} \text{mq}^5 \sigma ^5-1.28036\times 10^{23} \text{mq}^4 \sigma ^4+4.21972\times 10^{23} \text{mq}^3 \sigma
   ^3\nonumber\\  &-1.39031\times 10^{24} \text{mq}^2 \sigma ^2+4.06881\times 10^{24} \text{mq} \sigma +1.54586\times 10^{25})))
\end{align}

\begin{align}
    E_{\text{spin-spin}}^{3^1S_1} &=-\frac{1}{\text{mq}^{10}\sigma ^8 (\text{mq} \sigma )^{7/3}}
    (9.03103\times10^{-26}\alpha  Q^3 (9.92393\times 10^{23} \text{mq}^{10} \sigma ^{10}+5.93632\times
   10^{18} \text{mq}^8 Q^3 \sigma ^8 \nonumber\\  &(-189463. \text{mq}^2 \sigma ^2+215341. \text{mq} \sigma +1.01897\times 10^6)-2.36549\times
   10^{16} \text{mq}^6 Q^6 \sigma ^6 \nonumber\\  & (3.28318\times 10^7 \text{mq}^4 \sigma ^4-5.24566\times 10^7 \text{mq}^3 \sigma ^3+1.46201\times 10^8
   \text{mq}^2 \sigma ^2-3.99484\times 10^8 \text{mq} \sigma \nonumber\\  &-2.11336\times 10^9)+4.05188\times 10^{10} \text{mq}^4 Q^9 \sigma ^4
   (-7.3342\times 10^{12} \text{mq}^6 \sigma ^6+1.15509\times 10^{13} \text{mq}^5 \sigma ^5\nonumber\\  &-3.34333\times 10^{13} \text{mq}^4 \sigma
   ^4+1.18499\times 10^{14} \text{mq}^3 \sigma ^3-4.59266\times 10^{14} \text{mq}^2 \sigma ^2+1.58005\times 10^{15} \text{mq} \sigma
   \nonumber\\  &+8.44379\times 10^{15})+150503. \text{mq}^2 Q^{12} \sigma ^2 (-5.36115\times 10^{17} \text{mq}^8 \sigma ^8+8.17677\times 10^{17}
   \text{mq}^7 \sigma ^7\nonumber\\  &-2.34056\times 10^{18} \text{mq}^6 \sigma ^6+8.35122\times 10^{18} \text{mq}^5 \sigma ^5-3.3655\times 10^{19} \text{mq}^4
   \sigma ^4+1.4348\times 10^{20} \text{mq}^3 \sigma ^3\nonumber\\  &-6.22461\times 10^{20} \text{mq}^2 \sigma ^2+2.38049\times 10^{21} \text{mq} \sigma
   +1.27221\times 10^{22})+Q^{15} (-1.70714\times 10^{22} \text{mq}^{10} \sigma ^{10}\nonumber\\  &+2.52745\times 10^{22} \text{mq}^9 \sigma
   ^9-7.15156\times 10^{22} \text{mq}^8 \sigma ^8+2.54342\times 10^{23} \text{mq}^7 \sigma ^7-1.02616\times 10^{24} \text{mq}^6 \sigma
   ^6\nonumber\\  &+4.40896\times 10^{24} \text{mq}^5 \sigma ^5-1.97186\times 10^{25} \text{mq}^4 \sigma ^4+9.0234\times 10^{25} \text{mq}^3 \sigma
   ^3-4.13553\times 10^{26} \text{mq}^2 \sigma ^2\nonumber\\  &+1.68349\times 10^{27} \text{mq} \sigma +8.97715\times 10^{27}))
\end{align}

\begin{align}
    E_{\text{spin-spin}}^{4^1S_1} &=-\frac{1}{\text{mq}^{10}\sigma ^8 (\text{mq} \sigma )^{7/3}}
    (1.14545\times10^{-26}\alpha  Q^3 (8.67165\times 10^{24} \text{mq}^{10} \sigma ^{10}+3.09689\times
   10^{19} \text{mq}^8 Q^3 \sigma ^8 \nonumber\\  &(-317347. \text{mq}^2 \sigma ^2+443417. \text{mq} \sigma +2.57943\times 10^6)-1.00381\times
   10^{17} \text{mq}^6 Q^6 \sigma ^6 \nonumber\\  &(6.76034\times 10^7 \text{mq}^4 \sigma ^4-1.32814\times 10^8 \text{mq}^3 \sigma ^3+4.5535\times 10^8
   \text{mq}^2 \sigma ^2-1.54008\times 10^9 \text{mq} \sigma \nonumber\\  &-1.00512\times 10^{10})+9.25467\times 10^{10} \text{mq}^4 Q^9 \sigma ^4
   (-2.80919\times 10^{13} \text{mq}^6 \sigma ^6+5.44003\times 10^{13} \text{mq}^5 \sigma ^5\nonumber\\  &-1.93757\times 10^{14} \text{mq}^4 \sigma
   ^4+8.49813\times 10^{14} \text{mq}^3 \sigma ^3-4.05957\times 10^{15} \text{mq}^2 \sigma ^2+1.72438\times 10^{16} \text{mq} \sigma
   \nonumber\\  &+1.14194\times 10^{17})+227456. \text{mq}^2 Q^{12} \sigma ^2 (-3.10828\times 10^{18} \text{mq}^8 \sigma ^8+5.8259\times 10^{18}
   \text{mq}^7 \sigma ^7\nonumber\\  &-2.05163\times 10^{19} \text{mq}^6 \sigma ^6+9.05727\times 10^{19} \text{mq}^5 \sigma ^5-4.50048\times 10^{20}
   \text{mq}^4 \sigma ^4+2.3716\times 10^{21} \text{mq}^3 \sigma ^3\nonumber\\  &-1.27291\times 10^{22} \text{mq}^2 \sigma ^2+6.02666\times 10^{22} \text{mq}
   \sigma +4.00396\times 10^{23})+Q^{15} \nonumber\\  &(-1.49881\times 10^{23} \text{mq}^{10} \sigma ^{10}+2.72426\times 10^{23} \text{mq}^9 \sigma
   ^9-9.47937\times 10^{23} \text{mq}^8 \sigma ^8\nonumber\\  &+4.17045\times 10^{24} \text{mq}^7 \sigma ^7-2.07465\times 10^{25} \text{mq}^6 \sigma
   ^6+1.10195\times 10^{26} \text{mq}^5 \sigma ^5\nonumber\\  &-6.09984\times 10^{26} \text{mq}^4 \sigma ^4+3.46079\times 10^{27} \text{mq}^3 \sigma
   ^3-1.9683\times 10^{28} \text{mq}^2 \sigma ^2+9.94295\times 10^{28} \text{mq} \sigma\nonumber\\  & +6.60584\times 10^{29})))
\end{align}

Con estados con contribución de spin \(S = 1\), \(n ^3S_1 (\psi(nS))\),
\begin{align}
    E_{\text{spin-spin}}^{1^3S_1} &=-\frac{1}{\text{mq}^{10}\sigma ^8 (\text{mq} \sigma )^{7/3}}
    (1.62107\times10^{-22}\alpha  Q^3 (1.21048\times 10^{20} \text{mq}^{10} \sigma ^{10}+6.14565\times
   10^{15} \text{mq}^8 Q^3 \sigma ^8 \nonumber\\  &(-22322.7 \text{mq}^2 \sigma ^2+10745.6 \text{mq} \sigma +21535.2)-5.78217\times 10^{13}
   \text{mq}^6 Q^6 \sigma ^6 \nonumber\\  &(1.63952\times 10^6 \text{mq}^4 \sigma ^4-1.10301\times 10^6 \text{mq}^3 \sigma ^3+1.28017\times 10^6
   \text{mq}^2 \sigma ^2-1.20737\times 10^6 \text{mq} \sigma\nonumber\\  & -2.38805\times 10^6)+1.30371\times 10^9 \text{mq}^4 Q^9 \sigma ^4
   (-2.67681\times 10^{10} \text{mq}^6 \sigma ^6+1.7811\times 10^{10} \text{mq}^5 \sigma ^5\nonumber\\  &-2.12754\times 10^{10} \text{mq}^4 \sigma
   ^4+2.59831\times 10^{10} \text{mq}^3 \sigma ^3-3.86115\times 10^{10} \text{mq}^2 \sigma ^2+4.5408\times 10^{10} \text{mq} \sigma\nonumber\\  &
   +9.46711\times 10^{10})+26996.6 \text{mq}^2 Q^{12} \sigma ^2 (-3.37784\times 10^{14} \text{mq}^8 \sigma ^8+2.2086\times 10^{14}
   \text{mq}^7 \sigma ^7\nonumber\\  &-2.6238\times 10^{14} \text{mq}^6 \sigma ^6+3.23414\times 10^{14} \text{mq}^5 \sigma ^5-4.9628\times 10^{14} \text{mq}^4
   \sigma ^4+7.2159\times 10^{14} \text{mq}^3 \sigma ^3\nonumber\\  &-1.22789\times 10^{15} \text{mq}^2 \sigma ^2+1.58202\times 10^{15} \text{mq} \sigma
   +3.59258\times 10^{15})+Q^{15} \nonumber\\  &(-1.86673\times 10^{18} \text{mq}^{10} \sigma ^{10}+1.20543\times 10^{18} \text{mq}^9 \sigma
   ^9-1.42551\times 10^{18} \text{mq}^8 \sigma ^8\nonumber\\  &+1.75671\times 10^{18} \text{mq}^7 \sigma ^7-2.69936\times 10^{18} \text{mq}^6 \sigma
   ^6+3.954\times 10^{18} \text{mq}^5 \sigma ^5\nonumber\\  &-6.89901\times 10^{18} \text{mq}^4 \sigma ^4+1.07175\times 10^{19} \text{mq}^3 \sigma
   ^3-2.04905\times 10^{19} \text{mq}^2 \sigma ^2\nonumber\\  &+2.60098\times 10^{19} \text{mq} \sigma +7.03226\times 10^{19})))
\end{align}


\begin{align}
    E_{\text{spin-spin}}^{2^3S_1} &=-\frac{1}{\text{mq}^{10}\sigma ^8 (\text{mq} \sigma )^{7/3}}
    (6.07284\times10^{-25}\alpha  Q^3 (4.2386\times 10^{22} \text{mq}^{10} \sigma ^{10}+5.36656\times
   10^{17} \text{mq}^8 Q^3 \sigma ^8 \nonumber\\  &(-89512.5 \text{mq}^2 \sigma ^2+75337. \text{mq} \sigma +263977.)-2.88787\times 10^{15}
   \text{mq}^6 Q^6 \sigma ^6\nonumber\\  & (1.14872\times 10^7 \text{mq}^4 \sigma ^4-1.35812\times 10^7 \text{mq}^3 \sigma ^3+2.7973\times 10^7
   \text{mq}^2 \sigma ^2-5.53639\times 10^7 \text{mq} \sigma \nonumber\\  &-2.14383\times 10^8)+1.21828\times 10^{10} \text{mq}^4 Q^9 \sigma ^4
   (-1.03781\times 10^{12} \text{mq}^6 \sigma ^6+1.20967\times 10^{12} \text{mq}^5 \sigma ^5\nonumber\\  &-2.58479\times 10^{12} \text{mq}^4 \sigma
   ^4+6.64088\times 10^{12} \text{mq}^3 \sigma ^3-1.88944\times 10^{13} \text{mq}^2 \sigma ^2+4.74264\times 10^{13} \text{mq} \sigma\nonumber\\  &
   +1.83196\times 10^{14})+82525.9 \text{mq}^2 Q^{12} \sigma ^2 (-4.13939\times 10^{16} \text{mq}^8 \sigma ^8+4.68053\times 10^{16}
   \text{mq}^7 \sigma ^7\nonumber\\  &-9.89769\times 10^{16} \text{mq}^6 \sigma ^6+2.56099\times 10^{17} \text{mq}^5 \sigma ^5-7.56813\times 10^{17}
   \text{mq}^4 \sigma ^4+2.34627\times 10^{18} \text{mq}^3 \sigma ^3\nonumber\\  &-7.39225\times 10^{18} \text{mq}^2 \sigma ^2+2.04527\times 10^{19} \text{mq}
   \sigma +7.8157\times 10^{19})+Q^{15} \nonumber\\  &(-7.18381\times 10^{20} \text{mq}^{10} \sigma ^{10}+7.90922\times 10^{20} \text{mq}^9 \sigma
   ^9-1.65545\times 10^{21} \text{mq}^8 \sigma ^8\nonumber\\  &+4.27191\times 10^{21} \text{mq}^7 \sigma ^7-1.26385\times 10^{22} \text{mq}^6 \sigma
   ^6+3.94727\times 10^{22} \text{mq}^5 \sigma ^5\nonumber\\  &-1.28036\times 10^{23} \text{mq}^4 \sigma ^4+4.21972\times 10^{23} \text{mq}^3 \sigma
   ^3-1.39031\times 10^{24} \text{mq}^2 \sigma ^2\nonumber\\  &+4.06881\times 10^{24} \text{mq} \sigma +1.54586\times 10^{25})) )
\end{align}



\begin{align}
    E_{\text{spin-spin}}^{3^3S_1} &=-\frac{1}{\text{mq}^{10}\sigma ^8 (\text{mq} \sigma )^{7/3}}
    (3.01034\times10^{-26}\alpha  Q^3 (9.92393\times 10^{23} \text{mq}^{10} \sigma ^{10}+5.93632\times
   10^{18} \text{mq}^8 Q^3 \sigma ^8\nonumber\\  & (-189463. \text{mq}^2 \sigma ^2+215341. \text{mq} \sigma +1.01897\times 10^6)-2.36549\times
   10^{16} \text{mq}^6 Q^6 \sigma ^6 \nonumber\\  &(3.28318\times 10^7 \text{mq}^4 \sigma ^4-5.24566\times 10^7 \text{mq}^3 \sigma ^3+1.46201\times 10^8
   \text{mq}^2 \sigma ^2-3.99484\times 10^8 \text{mq} \sigma \nonumber\\  &-2.11336\times 10^9)+4.05188\times 10^{10} \text{mq}^4 Q^9 \sigma ^4
   (-7.3342\times 10^{12} \text{mq}^6 \sigma ^6+1.15509\times 10^{13} \text{mq}^5 \sigma ^5\nonumber\\  &-3.34333\times 10^{13} \text{mq}^4 \sigma
   ^4+1.18499\times 10^{14} \text{mq}^3 \sigma ^3-4.59266\times 10^{14} \text{mq}^2 \sigma ^2+1.58005\times 10^{15} \text{mq} \sigma\nonumber\\  &
   +8.44379\times 10^{15})+150503. \text{mq}^2 Q^{12} \sigma ^2 (-5.36115\times 10^{17} \text{mq}^8 \sigma ^8+8.17677\times 10^{17}
   \text{mq}^7 \sigma ^7\nonumber\\  &-2.34056\times 10^{18} \text{mq}^6 \sigma ^6+8.35122\times 10^{18} \text{mq}^5 \sigma ^5-3.3655\times 10^{19} \text{mq}^4
   \sigma ^4+1.4348\times 10^{20} \text{mq}^3 \sigma ^3\nonumber\\  &-6.22461\times 10^{20} \text{mq}^2 \sigma ^2+2.38049\times 10^{21} \text{mq} \sigma
   +1.27221\times 10^{22})+Q^{15}\nonumber\\  &(-1.70714\times 10^{22} \text{mq}^{10} \sigma ^{10}+2.52745\times 10^{22} \text{mq}^9 \sigma
   ^9-7.15156\times 10^{22} \text{mq}^8 \sigma ^8\nonumber\\  &+2.54342\times 10^{23} \text{mq}^7 \sigma ^7-1.02616\times 10^{24} \text{mq}^6 \sigma
   ^6+4.40896\times 10^{24} \text{mq}^5 \sigma ^5\nonumber\\  &-1.97186\times 10^{25} \text{mq}^4 \sigma ^4+9.0234\times 10^{25} \text{mq}^3 \sigma
   ^3-4.13553\times 10^{26} \text{mq}^2 \sigma ^2\nonumber\\  &+1.68349\times 10^{27} \text{mq} \sigma +8.97715\times 10^{27})))
\end{align}


\begin{align}
    E_{\text{spin-spin}}^{4^3S_1} &=-\frac{1}{\text{mq}^{10}\sigma ^8 (\text{mq} \sigma )^{7/3}}
    (3.81817\times10^{-27}\alpha  Q^3 (8.67165\times 10^{24} \text{mq}^{10} \sigma ^{10}+3.09689\times
   10^{19} \text{mq}^8 Q^3 \sigma ^8 \nonumber\\  &(-317347. \text{mq}^2 \sigma ^2+443417. \text{mq} \sigma +2.57943\times 10^6)-1.00381\times
   10^{17} \text{mq}^6 Q^6 \sigma ^6 \nonumber\\  &(6.76034\times 10^7 \text{mq}^4 \sigma ^4-1.32814\times 10^8 \text{mq}^3 \sigma ^3+4.5535\times 10^8
   \text{mq}^2 \sigma ^2-1.54008\times 10^9 \text{mq} \sigma \nonumber\\  &-1.00512\times 10^{10})+9.25467\times 10^{10} \text{mq}^4 Q^9 \sigma ^4
   (-2.80919\times 10^{13} \text{mq}^6 \sigma ^6+5.44003\times 10^{13} \text{mq}^5 \sigma ^5\nonumber\\  &-1.93757\times 10^{14} \text{mq}^4 \sigma
   ^4+8.49813\times 10^{14} \text{mq}^3 \sigma ^3-4.05957\times 10^{15} \text{mq}^2 \sigma ^2+1.72438\times 10^{16} \text{mq} \sigma\nonumber\\  &
   +1.14194\times 10^{17})+227456. \text{mq}^2 Q^{12} \sigma ^2 (-3.10828\times 10^{18} \text{mq}^8 \sigma ^8+5.8259\times 10^{18}
   \text{mq}^7 \sigma ^7\nonumber\\&-2.05163\times 10^{19} \text{mq}^6 \sigma ^6+9.05727\times 10^{19} \text{mq}^5 \sigma ^5-4.50048\times 10^{20}
   \text{mq}^4 \sigma ^4+2.3716\times 10^{21} \text{mq}^3 \sigma ^3\nonumber\\  &-1.27291\times 10^{22} \text{mq}^2 \sigma ^2+6.02666\times 10^{22} \text{mq}
   \sigma +4.00396\times 10^{23})+Q^{15} \nonumber\\  &(-1.49881\times 10^{23} \text{mq}^{10} \sigma ^{10}+2.72426\times 10^{23} \text{mq}^9 \sigma
   ^9-9.47937\times 10^{23} \text{mq}^8 \sigma ^8\nonumber\\  &+4.17045\times 10^{24} \text{mq}^7 \sigma ^7-2.07465\times 10^{25} \text{mq}^6 \sigma
   ^6+1.10195\times 10^{26} \text{mq}^5 \sigma ^5\nonumber\\  &-6.09984\times 10^{26} \text{mq}^4 \sigma ^4+3.46079\times 10^{27} \text{mq}^3 \sigma
   ^3-1.9683\times 10^{28} \text{mq}^2 \sigma ^2\nonumber\\  &+9.94295\times 10^{28} \text{mq} \sigma +6.60584\times 10^{29})))
\end{align}

\subsection{Estados P}
Para los estados P se encuentran todas las correcciones, pero se comienza
con la corrección spin-spin correspondiente a \(1^3P_j\), donde \(S = 1\).
\begin{align}
    E_{\text{spin-spin}}^{1^3P_j} &=-\frac{1}{\text{mq}^{2}(\text{mq} \sigma )^{7/3}}
    (3.9437\times10^{-17} \alpha  Q^3 (\frac{1.27806\times 10^{15} Q^{15}}{\text{mq}^{10} \sigma
   ^{10}}+\frac{2.96485\times 10^{14} Q^{15}}{\text{mq}^9 \sigma ^9}+\nonumber\\ &\frac{2520. (2.33288\times 10^9 Q^3+0.0139006 (3.53612\times
   10^{13}-7.5186\times 10^{12} Q^3)}{\text{mq}^8 \sigma ^8}+\nonumber\\ &\frac{2520(-1.63997\times 10^{10} (Q^3-5.)) Q^{12})}{\text{mq}^8 \sigma ^8}-\frac{19.7155
   (-6.71166\times 10^{12} Q^3-2.71143\times 10^{13}) Q^{12}}{\text{mq}^7 \sigma ^7}\nonumber\\ &+\frac{1.87383 (-4.62438\times 10^{13}
   Q^6-2.24746\times 10^{14} Q^3+8.51026\times 10^{14}) Q^9}{\text{mq}^6 \sigma ^6}\nonumber\\ &-\frac{3.00536 (-1.47879\times 10^{13}
   Q^6-7.32933\times 10^{13} Q^3-2.33659\times 10^{14}) Q^9}{\text{mq}^5 \sigma ^5}\nonumber\\ &+\frac{0.183626 (-1.5495\times 10^{14}
   Q^9-7.72184\times 10^{14} Q^6-2.96818\times 10^{15} Q^3+8.28867\times 10^{15}) Q^6}{\text{mq}^4 \sigma ^4}\nonumber\\ &-\frac{0.0261787
   (-6.22116\times 10^{14} Q^9-3.10425\times 10^{15} Q^6-1.22351\times 10^{16} Q^3-2.7886\times 10^{16}) Q^6}{\text{mq}^3 \sigma
   ^3}\nonumber\\ &+\frac{0.01344 (-8.27078\times 10^{14} Q^{12}-4.12919\times 10^{15} Q^9-1.63956\times 10^{16} Q^6-4.62012\times 10^{16}
   Q^3) Q^3}{\text{mq}^2 \sigma ^2}\nonumber\\ & + \frac{0.0134359Q^3(8.44885\times 10^{16})}{\text{mq}^2 \sigma ^2}+210. \text{mq}^2 (-3.70178\times 10^{10} Q^{15}-1.82697\times 10^{11}
   Q^{12}\nonumber\\ &-7.1398\times 10^{11} Q^9-2.04927\times 10^{12} Q^6-3.71123\times 10^{12} Q^3-2.68531\times 10^{12}) \sigma ^2-\nonumber\\ &\frac{0.0100563
   (-7.19265\times 10^{14} Q^{12}-3.59028\times 10^{15} Q^9-1.42753\times 10^{16} Q^6-4.15389\times 10^{16} Q^3)
   Q^3}{\text{mq} \sigma }\nonumber\\ &      + \frac{0.0100563(-5.70718\times 10^{16})}{\text{mq} \sigma }-0.0183954 \text{mq} (-2.70702\times 10^{14} Q^{15}-1.34679\times 10^{15} Q^{12}\nonumber\\ &-5.32934\times 10^{15}
   Q^9-1.55661\times 10^{16} Q^6-2.85029\times 10^{16} Q^3-1.47566\times 10^{16}) \sigma \nonumber\\ &+70. (-8.39004\times 10^{10}
   Q^{15}-4.18545\times 10^{11} Q^{12}-1.6638\times 10^{12} Q^9-4.88633\times 10^{12} Q^6\nonumber\\ &-8.69341\times 10^{12} Q^3+7.77171\times
   10^{12})))
\end{align}

La combinación de la corrección Tensorial y Spin-Orbita según estado,

\begin{align}
    E_{\text{T, S-P}}^{1^3P_0} =\frac{\text{mq}^3 \sigma ^2 (1.45688 \alpha -0.700179 \sigma )+\text{mq}^2 \sigma  (1.70317 \alpha +0.456281 \sigma )-0.439639 \text{mq} \sigma
   -0.6871}{\text{mq}^3 (\text{mq} \sigma )^{7/3}}
\end{align}

\begin{align}
    E_{\text{T, S-P}}^{1^3P_1} = \frac{\text{mq}^3 \sigma ^2 (0.582752 \alpha -0.350089 \sigma )+\text{mq}^2 \sigma  (0.681268 \alpha +0.228141 \sigma )-0.21982 \text{mq} \sigma
   -0.34355}{\text{mq}^3 (\text{mq} \sigma )^{7/3}}
\end{align}

\begin{align}
    E_{\text{T, S-P}}^{1^3P_2 } =\frac{\text{mq}^3 \sigma ^2 (0.350089 \sigma -0.641027 \alpha )+\text{mq}^2 \sigma  (-0.749395 \alpha -0.228141 \sigma )+0.21982 \text{mq} \sigma
   +0.34355}{\text{mq}^3 (\text{mq} \sigma )^{7/3}}
\end{align}

\end{appendix}