%\newpage
%\setcounter{page}{1}
\begin{center}
\begin{figure}
\centering%
\epsfig{file=Inicio/EscudoUN,scale=0.5}%
\end{figure}
\thispagestyle{empty} \vspace*{2.0cm} \textbf{\huge
Espectro del Charmonium y Bottomonium desde una perspectiva no relativista}\\[3.0cm]
\Large\textbf{Pedro José Leal Mesa}\\[5.0cm]
\small Universidad Nacional de Colombia\\
Facultad de Ciencias, Departamento de F\'{i}sica\\
Bogot\'{a}, Colombia\\
2022\\
\end{center}

\newpage{\pagestyle{empty}\cleardoublepage}

\newpage
\begin{center}
\thispagestyle{empty} \vspace*{0cm} \textbf{\huge
Espectro del Charmonium y Bottomonium desde una perspectiva no relativista}\\[2.0cm]
\Large\textbf{Pedro Jose Leal Mesa}\\[2.5cm]
\small Tesis o trabajo de grado presentada(o) como requisito parcial para optar al
t\'{\i}tulo de:\\
\textbf{ F\'{\i}sico}\\[2.5cm]
Directores:\\
Ph.D. Carlos Eduardo Sandoval Usme\\
Ph.D Diego Alejandro Milanés Carreño\\[2.0cm]
L\'{\i}nea de Investigaci\'{o}n:\\
Cromodin\'{a}mica cuántica (Quarkonia)\\[0.5cm]
Grupo de Investigaci\'{o}n:\\
FENYX\\[2.5cm]
Universidad Nacional de Colombia\\
Facultad de Ciencias, Departamento de F\'{i}sica\\
Bogot\'{a}, Colombia\\
2022\\
\end{center}

\newpage{\pagestyle{empty}\cleardoublepage}

\newpage
\thispagestyle{empty} \textbf{}\normalsize
\\\\\\%
\textbf{A mi Taita, a mis compañeros y a mis tutores por su paciencia.}\\[4.0cm]

\begin{flushright}
\begin{minipage}{8cm}
    \noindent
        \small
\textit{“Caminante, son tus huellas\\
el camino y nada más;\\
Caminante, no hay camino,\\
se hace camino al andar.\\
Al andar se hace el camino,\\
y al volver la vista atrás\\
se ve la senda que nunca\\
se ha de volver a pisar.\\
Caminante no hay camino\\
sino estelas en la mar.”\\  ― \textbf{Antonio Machado} }    
\end{minipage}
\end{flushright}
\newpage{\pagestyle{empty}\cleardoublepage}

\newpage
\thispagestyle{empty} \textbf{}\normalsize
\\\\\\%
\textbf{\LARGE Agradecimientos}
\addcontentsline{toc}{chapter}{\numberline{}Agradecimientos}\\\\
\\
A mi Padre, que no hay palabras para expresar todo el reconocimiento
y agradecimiento que poseo para con él, que siempre me ha apoyado
a seguir con mis metas.
\bigskip

A los profesores: Diego Milanés y Carlos Sandoval que, aunque hubo 
dificultades al principio para realizar el proyecto, siempre estuvieron dispuestos a  colaborar y acompañarme en este proceso, sin mencionar su gran paciencia.
\bigskip

Al profesor José Giraldo que confió en mis capacidades desde el primer semestre y por ello, hoy estoy presentando esta disertación.
\bigskip

A Pablo Figueroa y David Barón por su colaboración y consejos para la culminación de este trabajo.
\bigskip

A mis compañeros Ronald Cortés, Catalina Moreno, Brayan Castiblanco, Francisco Ortiz, Camilo Prada y Daniel Ladino que estuvieron a mi lado, me ayudaron y trabajaron conmigo durante esta etapa de mi vida.

\begin{flushright}
\begin{minipage}{8cm}
    \noindent
        \small
\textbf{Con cariño \\ Pedro José Leal Mesa}    
\end{minipage}
\end{flushright}


\newpage{\pagestyle{empty}\cleardoublepage}



\textbf{\LARGE Resumen}
\addcontentsline{toc}{}{\numberline{}Resumen}\\\\

El comportamiento de los mesones pesados ha sido un desafío a las limitaciones de la teoría QCD en este rango de energía. Por ello, se propone el uso de un potencial con gran aceptación en la comunidad científica, como es el potencial de Cornell que se deduce de las contribuciones del intercambio gluónico. Este potencial se toma dentro de la ecuación de Schrödinger, que se resuelve por límites, en el caso \(r \to \infty\) su solución depende de la función de Airy, que permite encontrar una energía estado base que se usa para una minimización \(\chi^2\) basado en los datos experimentales, con el fin de ajustar las masas de los estados S. Posteriormente se agregan las correcciones espinoriales para realizar un ajuste con diferentes estados (\(l > 0\)). Los estados calculados se aproximan  bien a los estados teóricos y experimentales.\\\\

\textbf{\LARGE Abstract}\\\\

The behavior of heavy mesons has been a challenge to the limitations of QCD theory in this energy range. For this reason, the use of a potential with great acceptance in the scientific community is proposed, such as the Cornell potential which is deducted from the contributions of the gluonic exchange. This potential is taken into the Schrödinger equation, which is solved by limits, in the case \(r \to \infty\) its solution depends on the Airy function, which allows finding a ground state energy that is used for a minimization \(\chi^2\) based on the experimental data, to fit the masses of the S states. Subsequently, spinorial corrections are added to perform a fit with different states (\(l > 0\)). The calculated states approximate well to the theoretical and experimental states.\\


