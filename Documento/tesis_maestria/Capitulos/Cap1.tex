% !TeX root = ../main.tex
\chapter{Introducción y Objetivos}
\label{cap:introduccion}

En este primer capítulo se describe el contexto general de la investigación, enfocándose en la importancia de la cuantificación de la incertidumbre en el análisis de series temporales. Se presenta el planteamiento del problema, los retos actuales de la predicción probabilística y la justificación académica de este trabajo. Finalmente, se definen los objetivos general y específicos que guiarán el desarrollo de la tesis y se presenta la estructura general del documento.

\section{Introducción}

La predicción es una tarea fundamental en diversas disciplinas, siendo de particular interés en el análisis de series de tiempo. Tradicionalmente, la literatura y la práctica se han centrado en estimaciones puntuales, como la media o el valor esperado de una variable futura. Sin embargo, este enfoque omite información crucial sobre la incertidumbre asociada a la predicción. Problemas clásicos, como la optimización de inventarios o la gestión de riesgos financieros, ilustran la necesidad de ir más allá de la media y considerar la distribución completa de los posibles resultados futuros para una toma de decisiones óptima.

La predicción probabilística aborda esta necesidad proporcionando no solo un valor central, sino una distribución de probabilidad o un conjunto de cuantiles sobre los valores futuros. En este contexto, surge la teoría de la \textit{Predicción Conformal} (Conformal Prediction - CP) \parencite{Vovk2005} como un marco robusto para la construcción de regiones de predicción con garantías de cobertura válidas. Sin embargo, es importante destacar que la CP fue desarrollada originalmente para datos idéntica e independientemente distribuidos (i.i.d.), es decir, observaciones sin ningún tipo de correlación temporal. Además, en su concepción inicial, la CP se enfoca principalmente en la construcción de intervalos de predicción en lugar de distribuciones predictivas completas. A diferencia de otros métodos, la CP posee validez exacta bajo los supuestos de intercambiabilidad y es aplicable a diversos modelos predictivos, lo que la convierte en una alternativa metodológica prometedora para extender hacia el pronóstico probabilístico en series temporales.

La relevancia comparativa de los métodos que se analizarán en este trabajo radica en sus características distintivas: mientras que la CP ofrece garantías teóricas de cobertura bajo supuestos mínimos en escenarios i.i.d., los métodos de remuestreo como el \textit{Bootstrapping} \parencite{Lahiri2003} proporcionan flexibilidad no paramétrica para capturar la estructura de dependencia temporal, y los modelos de aprendizaje profundo como \textit{DeepAR} \parencite{Salinas2020} destacan por su capacidad de modelar dependencias temporales complejas en múltiples series simultáneamente. Esta diversidad metodológica permite una evaluación integral de las fortalezas y limitaciones de cada enfoque.

La aplicación de CP a series de tiempo presenta desafíos críticos, ya que la dependencia temporal y los desplazamientos distribucionales suelen invalidar el supuesto de intercambiabilidad. Además, la transición de CP desde la generación de intervalos hacia distribuciones predictivas completas sigue siendo un área poco explorada. Este trabajo propone adaptar y extender estos métodos para manejar la no-intercambiabilidad y permitir la predicción distribucional, contrastando su desempeño con técnicas establecidas. Con ello, se busca cerrar una brecha relevante en la literatura: la falta de una evaluación sistemática de CP aplicada a series temporales para la obtención de distribuciones predictivas integrales.

\section{Planteamiento del problema}

La predicción en series de tiempo enfrenta retos fundamentales que motivan esta investigación. En primer lugar, la naturaleza secuencial de los datos introduce dependencia temporal que viola el supuesto de independencia fundamental en muchos métodos estadísticos estándar, incluyendo la predicción conformal en su formulación original. Esta dependencia temporal implica que las observaciones consecutivas están correlacionadas, lo que afecta tanto la validez de las garantías teóricas como la eficiencia de los métodos de cuantificación de incertidumbre. En segundo lugar, las características de los datos pueden cambiar con el tiempo a través de tendencias, estacionalidad o cambios estructurales, manifestando no estacionariedad que complica el modelado a largo plazo y desafía los supuestos de intercambiabilidad requeridos por métodos como la predicción conformal. Finalmente, generar pronósticos probabilísticos que sean simultáneamente precisos (bien calibrados) e informativos (eficientes o nítidos) representa una tarea compleja, especialmente cuando se busca construir distribuciones predictivas completas en lugar de simples intervalos de predicción.

Los métodos actuales presentan limitaciones que este estudio busca contrastar. Por un lado, los modelos clásicos de tipo autorregresivo dependen fuertemente de supuestos sobre la estructura del proceso y la distribución del ruido (usualmente Gaussiano). Por otro lado, técnicas de remuestreo como el \textit{Bootstrapping} \parencite{Lahiri2003}, aunque flexibles para capturar dependencia temporal, pueden tener dificultades cerca de límites de no estacionariedad. Finalmente, modelos modernos de aprendizaje profundo como \textit{DeepAR} \parencite{Salinas2020} ofrecen un alto poder predictivo y la capacidad de generar distribuciones predictivas completas, pero a menudo operan como ``cajas negras'' y carecen de las garantías teóricas formales de cobertura que ofrece el enfoque conformal bajo condiciones de intercambiabilidad.

El problema central de este trabajo se resume en la siguiente pregunta: \textbf{¿Cómo adaptar y evaluar un modelo de predicción probabilística basado en la teoría conformal para series de tiempo que considere la dependencia temporal y permita la construcción de distribuciones predictivas completas, y cuál es su desempeño comparativo frente a métodos de referencia como Bootstrapping y DeepAR?}

\section{Justificación}

La cuantificación precisa de la incertidumbre es esencial para la toma de decisiones informada en áreas como la economía, meteorología y planificación de recursos. Adaptar la Predicción Conformal al dominio temporal representa un avance metodológico importante, ya que ofrece garantías de cobertura en muestra finita y una aplicabilidad general (libre de distribución) que originalmente fueron desarrolladas para el caso i.i.d.

A pesar de los desarrollos recientes en predicción conformal, existe un vacío notable en la literatura respecto a su aplicación sistemática en series temporales con características de no intercambiabilidad. La brecha metodológica que este trabajo busca cerrar es doble: por un lado, la mayoría de los trabajos en CP se han centrado en la construcción de intervalos de predicción bajo supuestos de intercambiabilidad, relegando el desarrollo de métodos para predicción distribucional completa en contextos temporales. Por otro lado, la comparación rigurosa de CP adaptada mediante ponderación temporal frente a métodos establecidos y de vanguardia utilizando métricas probabilísticas apropiadas ha sido escasamente explorada. En particular, la evaluación mediante el promedio del \textit{Continuous Ranked Probability Score} (ECRPS) y pruebas de significancia estadística como el test de Diebold-Mariano permitirán establecer comparaciones robustas entre las predicciones de los métodos.

La comparación sistemática con métodos estándar (Bootstrapping), estado del arte (\textit{DeepAR}) y otros métodos proporcionará una perspectiva clara sobre las fortalezas y debilidades del enfoque conformal en escenarios reales y simulados, facilitando la selección informada de métodos de pronóstico probabilístico en aplicaciones prácticas.

\section{Objetivos}

\subsection{Objetivo General}
Formular un modelo fundamentado en la teoría conformal para realizar predicciones probabilísticas en el contexto de series de tiempo, evaluando su desempeño mediante una comparación con modelos establecidos como Bootstrapping y DeepAR.

\subsection{Objetivos Específicos}

\begin{itemize}
    \item Desarrollar un modelo de predicción probabilística que integre la teoría de predicción conformal con técnicas de modelado de series temporales.
    
    \item Diseñar un entorno de simulación para evaluar el comportamiento del modelo propuesto en diversos escenarios de series temporales, incorporando estructuras temporales y condiciones de ruido variadas.
    
    \item Comparar el desempeño del modelo propuesto con modelos de referencia como DeepAR y Bootstrapping en cada escenario simulado, utilizando métricas enfocadas en predicciones probabilísticas.
    
    \item Implementar la metodología en un caso de estudio con datos reales, cuantificando su desempeño y relevancia práctica para aplicaciones de pronóstico.
\end{itemize}

\section{Estructura de la tesis}

El presente documento se organiza de la siguiente manera:

\textbf{Capítulo 2: Predicción Conformal.} Se presenta una introducción al pronóstico probabilístico, se explican las principales métricas para evaluar el desempeño de los modelos (incluyendo CRPS, calibración y nitidez), se desarrollan los fundamentos de los sistemas de predicción conformal, y se incluye una discusión sobre la extensión de la demostración de consistencia universal en escenarios estacionarios y ergódicos.

\textbf{Capítulo 3: Metodología de Simulación.} Se detalla el diseño del entorno de simulación, incluyendo la definición de los procesos generadores de datos, la formulación de los modelos predictivos, los cambios necesarios para su aplicación en este trabajo, y las extensiones de las simulaciones para diferentes escenarios.

\textbf{Capítulo 4: Resultados de Simulación.} Se exponen y analizan los hallazgos derivados del marco experimental descrito en el capítulo anterior. El análisis incluye la comparación del desempeño mediante la métrica ECRPS, el examen del comportamiento de modelos específicos ante diversos escenarios y la validación de los resultados a través de pruebas de significancia estadística empleando el test de Diebold-Mariano.

\textbf{Capítulo 5: Aplicaciones a Datos Reales.} Se implementan los métodos desarrollados en casos de estudio con datos reales, se presentan los resultados obtenidos y se discute su relevancia práctica.

\textbf{Capítulo 6: Conclusiones.} Se sintetizan los hallazgos principales del trabajo, se discuten las limitaciones del estudio, y se proponen direcciones para investigación futura.