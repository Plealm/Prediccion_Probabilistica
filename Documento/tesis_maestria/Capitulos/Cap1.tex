% !TeX root = ../main.tex
\chapter{Introducción y Objetivos}
\label{cap:introduccion}

En este capítulo se describe el contexto general de la investigación, enfocándose en la importancia de la cuantificación de la incertidumbre en el análisis de series temporales. Se presenta el planteamiento del problema, los retos de extender la predicción conformal desde el contexto i.i.d. hacia series temporales, y la justificación académica de este trabajo. Finalmente, se definen los objetivos general y específicos que guiaron el desarrollo de la tesis y se presenta la estructura general del documento.

\section{Introducción}

La predicción es una tarea fundamental en diversas disciplinas, siendo de particular interés en el análisis de series de tiempo. Tradicionalmente, la literatura y la práctica se han centrado en estimaciones puntuales, como la media o el valor esperado de una variable futura. Sin embargo, este enfoque omite información crucial sobre la incertidumbre asociada a la predicción. Problemas clásicos, como la optimización de inventarios o la gestión de riesgos financieros, ilustran la necesidad de ir más allá de la media y considerar la distribución completa de los posibles resultados futuros para una toma de decisiones óptima.

La predicción probabilística aborda esta necesidad proporcionando no solo un valor central, sino una distribución de probabilidad o un conjunto de cuantiles sobre los valores futuros. Para datos idéntica e independientemente distribuidos (i.i.d.), uno de los marcos teóricos más robustos es la \textit{Predicción Conformal} (Conformal Prediction - CP) \parencite{Vovk2005}. La CP ofrece garantías exactas de cobertura bajo supuestos de intercambiabilidad y es aplicable a diversos modelos predictivos subyacentes. Su principal fortaleza radica en que es libre de distribución: no requiere supuestos paramétricos sobre la forma de las distribuciones y proporciona regiones de predicción con cobertura controlada en muestra finita.

Sin embargo, la CP presenta dos limitaciones fundamentales que motivan esta investigación. Primero, fue desarrollada originalmente para el contexto i.i.d., donde las observaciones no presentan correlación temporal. Segundo, su formulación clásica se enfoca en la construcción de intervalos o regiones de predicción, sin proporcionar distribuciones predictivas completas. Esta tesis aborda precisamente estas dos extensiones: Adaptar la CP desde datos i.i.d. hacia series temporales con dependencia temporal, y expandir su aplicación desde intervalos de predicción hacia la generación de distribuciones predictivas integrales.

La aplicación de CP a series de tiempo presenta desafíos críticos. La dependencia temporal y los posibles desplazamientos distribucionales invalidan el supuesto de intercambiabilidad, base teórica de las garantías de cobertura de CP. Además, mientras que métodos establecidos como el \textit{Bootstrapping} \parencite{Lahiri2003} han sido adaptados para capturar estructuras de dependencia temporal mediante remuestreo no paramétrico, y técnicas modernas de aprendizaje profundo como \textit{DeepAR} \parencite{Salinas2020} modelan dependencias complejas generando distribuciones predictivas automáticamente, la extensión sistemática de CP a este dominio sigue siendo un área poco explorada.

Este trabajo propone adaptar la CP mediante esquemas de ponderación temporal para manejar la no-intercambiabilidad, extenderla hacia la predicción distribucional completa, y evaluar su desempeño comparativo frente a estos métodos de referencia utilizando métricas probabilísticas rigurosas. Con ello, se busca cerrar una brecha relevante en la literatura: la falta de un marco metodológico y de una evaluación sistemática de CP aplicada a series temporales para la obtención de distribuciones predictivas.

\section{Planteamiento del problema}

La predicción en series de tiempo enfrenta retos fundamentales. La naturaleza secuencial de los datos introduce dependencia temporal que viola el supuesto de independencia requerido por la CP en su formulación original. Esta correlación entre observaciones consecutivas afecta tanto la validez de las garantías teóricas como la eficiencia de los métodos de cuantificación de incertidumbre. Adicionalmente, las características de los datos pueden cambiar con el tiempo a través de tendencias, estacionalidad o cambios estructurales, manifestando no estacionariedad que desafía directamente el supuesto de intercambiabilidad de la predicción conformal.

Los métodos actuales presentan compensaciones importantes. Los modelos autorregresivos clásicos dependen de supuestos paramétricos sobre la estructura del proceso y la distribución del ruido. El \textit{Sieve Bootstrapping} \parencite{Lahiri2003}, aunque flexible para capturar dependencia temporal, puede tener dificultades cerca de límites de no estacionariedad. Los modelos de aprendizaje profundo como \textit{DeepAR} \parencite{Salinas2020} ofrecen alto poder predictivo y generan distribuciones predictivas completas, pero operan como ``cajas negras'' y carecen de garantías teóricas formales de cobertura.

El problema central de este trabajo se resume en la siguiente pregunta: ¿cómo adaptar la teoría de predicción conformal desde el contexto i.i.d. hacia series temporales con dependencia temporal, extendiendo su aplicación desde intervalos hacia distribuciones predictivas completas, y cuál es su desempeño comparativo frente a métodos establecidos como Bootstrapping y DeepAR?

\section{Justificación}

La cuantificación precisa de la incertidumbre es esencial para la toma de decisiones informada en áreas como economía, meteorología y planificación de recursos. La Predicción Conformal ofrece garantías de cobertura en muestra finita y aplicabilidad general (libre de distribución) originalmente desarrolladas para el caso i.i.d. Extender estas propiedades al dominio temporal representa un avance metodológico importante.

Existe un vacío notable en la literatura respecto a la aplicación sistemática de CP en series temporales con no-intercambiabilidad. La brecha metodológica que este trabajo busca cerrar es doble. Primero, la mayoría de trabajos en CP se han centrado en intervalos de predicción bajo intercambiabilidad, relegando el desarrollo de métodos para predicción distribucional completa en contextos temporales. Segundo, la comparación rigurosa de CP adaptada mediante ponderación temporal frente a métodos establecidos utilizando métricas probabilísticas apropiadas ha sido escasamente explorada.

La evaluación mediante el promedio del \textit{Continuous Ranked Probability Score} (ECRPS) metrica que se desarrollará en la sección \ref{subsec:ecrps}, además pruebas de significancia estadística como el test de Diebold-Mariano permitirán establecer comparaciones robustas. Esta comparación sistemática con métodos estándar (Sieve Bootstrapping) y estado del arte (\textit{DeepAR}) proporcionará una perspectiva clara sobre las fortalezas y debilidades del enfoque conformal en escenarios reales y simulados, facilitando la selección informada de métodos de pronóstico probabilístico en aplicaciones prácticas.
\section{Objetivos}

\subsection{Objetivo General}
Formular una metodología fundamentada en la teoría conformal para realizar predicciones probabilísticas en el contexto de series de tiempo, evaluando su desempeño mediante una comparación con modelos establecidos como Bootstrapping y DeepAR.

\subsection{Objetivos Específicos}

\begin{itemize}
    \item Desarrollar un modelo de predicción probabilística que integre la teoría de predicción conformal con técnicas de modelado de series temporales.
    
    \item Diseñar un entorno de simulación para evaluar el comportamiento del modelo propuesto en diversos escenarios de series temporales, incorporando estructuras temporales y condiciones de ruido variadas.
    
    \item Comparar el desempeño del modelo propuesto con modelos de referencia como DeepAR y Sieve Bootstrapping en cada escenario simulado, utilizando métricas enfocadas en predicciones probabilísticas.
    
    \item Implementar la metodología en un caso de estudio con datos reales, cuantificando su desempeño y relevancia práctica para aplicaciones de pronóstico.
\end{itemize}

\section{Estructura de la tesis}

El presente documento se organiza de la siguiente manera:

\textbf{Capítulo 2: Sistemas de Predicción Conformal.} Se presenta una introducción al pronóstico probabilístico, se explican las principales métricas para evaluar el desempeño de los modelos (incluyendo CRPS, calibración y nitidez), se desarrollan los fundamentos de los sistemas de predicción conformal, y se incluye una discusión sobre la extensión de la demostración de consistencia universal en escenarios estacionarios y ergódicos.

\textbf{Capítulo 3: Metodología de Simulaciones.} Se detalla el diseño del entorno de simulación, incluyendo la definición de los procesos generadores de datos, la formulación de los modelos predictivos, los cambios necesarios para su aplicación en este trabajo, y las extensiones de las simulaciones para diferentes escenarios.

\textbf{Capítulo 4: Resultados y Análisis de Simulaciones.} Se exponen y analizan los hallazgos derivados del marco experimental descrito en el capítulo anterior. El análisis incluye la comparación del desempeño mediante la métrica ECRPS, el examen del comportamiento de modelos específicos ante diversos escenarios y la validación de los resultados a través de pruebas de significancia estadística empleando el test de Diebold-Mariano.

\textbf{Capítulo 5: Aplicaciones a Series de Tiempo Reales.} Se implementan los métodos desarrollados en casos de estudio con datos reales, se presentan los resultados obtenidos y se discute su relevancia práctica.

\textbf{Capítulo 6: Conclusiones.} Se sintetizan los hallazgos principales del trabajo, se discuten las limitaciones del estudio, y se proponen direcciones para investigación futura.